%%%%%%%%%%%%%%%%%%%%%%%%%%%%%%%%%%%%%%%%%%%%%%%%%%%%%%%%%%%%%%%%%%%%%%%
% Title: LaTeX Thesis Template Usage Guide Chapter
% Purpose: Comprehensive guide for using the thesis template
% Compiler: pdflatex
% OS: Manjaro 
% Version: V1
% Written on: July 07, 2025
% Revision Date: July 07, 2025
% Author: Kingsuk Majumdar
% Copyright (c) 2025 Kingsuk Majumdar
%%%%%%%%%%%%%%%%%%%%%%%%%%%%%%%%%%%%%%%%%%%%%%%%%%%%%%%%%%%%%%%%%%%%%%%

\chapter{LaTeX Thesis Template Usage Guide}
\label{ch:template_guide}
\justifying

\section{Introduction}
\label{sec:intro}

The main aim of this chapter is to provide a comprehensive guide for using the LaTeX thesis template specifically designed for Dr. B. C. Roy Engineering College. This template has been developed to streamline the thesis writing process for both undergraduate and postgraduate students while maintaining institutional formatting standards and academic presentation quality.

The template architecture follows a modular approach with clear separation between user inputs and system-level formatting commands. The primary advantage of this template lies in its automated handling of multi-student configurations, conditional rendering of content based on degree type, and professional formatting that adheres to institutional guidelines.

\section{Template Architecture and Directory Structure}
\label{sec:architecture}

The template follows a well-organized hierarchical structure that facilitates easy content management and compilation. The complete directory structure is presented below:

\begin{verbatim}
	ug-thesis-template/
	|-- main.tex                    # Main document file (User Input Section)
	|-- thesis.cls                  # LaTeX class file (Formatting Engine)
	|-- references.bib              # Bibliography database
	|-- mcode.sty                   # MATLAB code highlighting package
	|-- README.md                   # Documentation file
	|-- LICENSE.lic                 # License information
	|-- Frontmatter/
	|   |-- Declaration.tex         # Student declaration page (Dont Change it) 
	|   |-- Certificate.tex   # Supervisor approval   certificate (Dont Change it)
	|   |-- Acknowledgment.tex      # Acknowledgments section
	|   |-- Abstract.tex           # Abstract and keywords
	|   +-- Acronyms.tex           # List of abbreviations and nomenclature
	|-- Chapters/
	|   |-- Chapter01_Introduction.tex    # Introduction  chapter (MUST BE)
	|   |-- Chapter02_Literature.tex     # Literature review (MUST BE)
	|   |-- Chapter02_Table.tex          # Table examples
	|   |-- Chapter03_Figure.tex         # Figure examples
	|   |-- Chapter04_Math.tex           # Mathematical expressions
	|   |-- Chapter03_Methodology.tex    # Research methodology
	|   |-- Chapter04_Implementation.tex # Implementation details
	|   |-- Chapter05_Results.tex        # Results and analysis (MUST BE)
	|   +-- Chapter06_Conclusion.tex     # Conclusions and future work (MUST BE)
	|-- Backmatter/
	|   |-- PublicationsList.tex    # Publications by authors
	|   +-- AuthorBio.tex          # Author biographies (Strictly PG/PhD only)
	|-- Figures/
	|   |-- college_logo.png       # Institutional logo (required)
	|   |-- StudentOne_photo.jpg   # Student photograph
	|   |-- StudentTwo_photo.jpg   # Student photograph
	|   |-- StudentThree_photo.jpg # Student photograph
	|   |-- Chapter01/             # Chapter-wise figure organization
	|   |-- Chapter02/
	|   |-- Chapter03/
	|   |-- Chapter04/
	|   |-- Chapter05/
	|   +-- Chapter06/
	+-- OUTPUT/                    # Generated output files (after compilation)
	|-- main.pdf              # Final thesis document
	|-- main.aux              # Auxiliary file
	|-- main.bbl              # Bibliography file
	|-- main.blg              # Bibliography log
	|-- main.log              # Compilation log
	|-- main.toc              # Table of contents
	|-- main.lof              # List of figures
	+-- main.lot              # List of tables
\end{verbatim}

\section{Configuration and User Input Section}
\label{sec:configuration}

The template utilizes a sophisticated variable definition system within the \texttt{main.tex} file. All user-specific information is contained within the clearly marked ``USER INPUT SECTION'' which must be modified according to individual thesis requirements.

\subsection{Thesis Information Configuration}
\label{subsec:thesis_info}

The fundamental thesis parameters are defined through the following commands:

\begin{verbatim}
	%% Thesis Information
	\ThesisTitle{Long Thesis Title}
	\ShortTitle{Short Thesis Title}
	\Department{Department of Electrical Engineering}
	\College{Dr. B. C. Roy Engineering College}
	\University{Maulana Abul Kalam Azad University of Technology, West Bengal}
	\DegreeType{Bachelor of Technology (B. TECH)}
	\ThesisYear{2025}
	\ThesisMonth{May}
	\Location{Durgapur}
	\AY{2024-2025}
	\Address{Durgapur – 713206, West Bengal, India}
	\Affiliation{(An Autonomous Institute, Affiliated To MAKAUT)}
\end{verbatim}

\subsection{Project-Specific Information}
\label{subsec:project_info}

For academic projects, the following parameters must be configured:

\begin{verbatim}
	%% Project Information
	\GroupNo{Group 00}
	\PaperName{Final Year Project Stage-II}
	\PaperCode{PWEE881}
\end{verbatim}

\subsection{Student Configuration System}
\label{subsec:student_config}

The template implements a dynamic student handling system that automatically adjusts content based on the number of students specified:

\begin{verbatim}
	%% Number of Students Configuration
	\NumberOfStudents{3}  % Range: 1-3 for UG, 1 for PG
	
	%% Student Information
	\StudentOne{Pradosh Chandra Mitter}
	\RollOne{18/EE/045}
	\RegOne{184410301045}
	\EmailOne{pradosh.mitter@bcrec.ac.in}
	\PhotoOne{Figures/StudentOne_photo.jpg}
	
	\StudentTwo{Tapesh Ranjan Mitter}
	\RollTwo{18/EE/052}
	\RegTwo{184410301052}
	\EmailTwo{tapesh.mitter@bcrec.ac.in}
	\PhotoTwo{Figures/StudentTwo_photo.jpg}
	
	\StudentThree{Lalmohan Gonguly}
	\RollThree{18/EE/063}
	\RegThree{184410301063}
	\EmailThree{lalmohan.gonguly@bcrec.ac.in}
	\PhotoThree{Figures/StudentThree_photo.jpg}
\end{verbatim}

\subsection{Supervision Structure}
\label{subsec:supervision}

The template accommodates both single supervisor and co-supervisor configurations:

\begin{verbatim}
	%% Supervisor Configuration
	\HasCoSupervisor{2} % 1=supervisor only, 2=both supervisor and co-supervisor
	\Supervisor{Professor C.V. Raman}
	\SupervisorDesignation{Professor}
	\SupervisorEmail{cv.raman@bcrec.ac.in}
	\SupervisorDept{Department of Electrical Engineering}
	
	\CoSupervisor{Acharya Prafulla Chandra Ray}
	\CoSupervisorDesignation{Assistant Professor}
	\CoSupervisorEmail{pc.ray@bcrec.ac.in}
	\CoSupervisorDept{Department of Electrical Engineering}
\end{verbatim}

\section{Degree-Specific Configurations}
\label{sec:degree_config}

\subsection{Undergraduate (UG) Thesis Requirements}
\label{subsec:ug_requirements}

For undergraduate theses, the following specifications must be observed:

\begin{itemize}
	\item \textbf{Maximum Students}: 3 students per group
	\item \textbf{Author Biography}: Not included in final document
	\item \textbf{Degree Type}: Bachelor of Technology (B. TECH)
	\item \textbf{Paper Code}: PWEE881 (Final Year Project Stage-II)
\end{itemize}

The configuration for undergraduate thesis should exclude author biography by commenting out the relevant include statement:

\begin{verbatim}
	% Publications by authors
	%%%%%%%%%%%%%%%%%%%%%%%%%%%%%%%%%%%%%%%%%%%%%%%%%%%%%%%%%%%%%%%%%%%%%%%
% Title: Publications by the Authors
% Purpose: List of publications by thesis authors
% Compiler: pdflatex
% OS: Manjaro 
% Version: V1
% Written on: July 05, 2025
% Revision Date: July 05, 2025
% Author: Kingsuk Majumdar
% Copyright (c) 2025 Kingsuk Majumdar
%%%%%%%%%%%%%%%%%%%%%%%%%%%%%%%%%%%%%%%%%%%%%%%%%%%%%%%%%%%%%%%%%%%%%%%

\chapter*{Publications by the Authors}
\addcontentsline{toc}{chapter}{Publications by the Authors}
\chaptermark{Publications by the Authors}
\thispagestyle{plain}

\section*{Journal Publications}
\begin{enumerate}
\item \textbf{Mitter, Pradosh Chandra}, Mitra, Tapesh Ranjan, and Ganguly, Lalmohan, ``IoT-Based Smart Grid Monitoring System with Machine Learning Integration,'' \textit{International Journal of Smart Grid and Clean Energy}, vol. 14, no. 2, pp. 45-58, 2025. DOI: 10.12720/sgce.14.2.45-58

\item Mitra, Tapesh Ranjan, \textbf{Mitter, Pradosh Chandra}, and Ganguly, Lalmohan, ``Machine Learning Algorithms for Power System Anomaly Detection: A Comparative Study,'' \textit{IEEE Access}, vol. 13, pp. 15234-15247, 2025. DOI: 10.1109/ACCESS.2025.3456789

\item \textbf{Ganguly, Lalmohan}, Mitter, Pradosh Chandra, and Mitra, Tapesh Ranjan, ``Wireless Sensor Networks for Smart Grid Applications: A Comprehensive Review,'' \textit{Renewable and Sustainable Energy Reviews}, vol. 145, article 111098, 2025. DOI: 10.1016/j.rser.2025.111098
\end{enumerate}

\section*{Conference Publications}
\begin{enumerate}
\item \textbf{Ganguly, Lalmohan}, Mitter, Pradosh Chandra, and Mitra, Tapesh Ranjan, ``Development of Wireless Sensor Network for Real-time Grid Monitoring,'' in \textit{Proceedings of IEEE International Conference on Power Electronics and Drives}, New Delhi, India, March 2025, pp. 234-239. DOI: 10.1109/IPED.2025.9123456

\item Mitter, Pradosh Chandra, \textbf{Mitra, Tapesh Ranjan}, and Ganguly, Lalmohan, ``Performance Evaluation of ML Algorithms in Smart Grid Applications,'' in \textit{National Conference on Advances in Electrical Engineering}, Dr. B. C. Roy Engineering College, Durgapur, February 2025, pp. 67-72.

\item \textbf{Mitra, Tapesh Ranjan}, Ganguly, Lalmohan, and Mitter, Pradosh Chandra, ``IoT Security Framework for Smart Grid Systems,'' in \textit{International Conference on Cybersecurity in Smart Grid}, IIT Kharagpur, January 2025, pp. 156-161.
\end{enumerate}

\section*{Under Review}
\begin{enumerate}
\item \textbf{Mitter, Pradosh Chandra}, Mitra, Tapesh Ranjan, Ganguly, Lalmohan, and Majumdar, Kingsuk, ``Comprehensive Analysis of IoT Security in Smart Grid Systems,'' \textit{Renewable and Sustainable Energy Reviews}, Elsevier. [Under Review - Submitted December 2024]

\item Ganguly, Lalmohan, \textbf{Mitter, Pradosh Chandra}, and Mitra, Tapesh Ranjan, ``Edge Computing for Real-time Smart Grid Data Processing,'' \textit{IEEE Transactions on Smart Grid}, IEEE. [Under Review - Submitted January 2025]
\end{enumerate}

\cleardoublepage
	
	% About the authors - COMMENTED OUT FOR UG
	%\include{Backmatter/AuthorBio} % Applicable for PG/PhD ONLY
\end{verbatim}

\subsection{Postgraduate (PG) Thesis Requirements}
\label{subsec:pg_requirements}

For postgraduate theses, the specifications are:

\begin{itemize}
	\item \textbf{Number of Students}: 1 student only
	\item \textbf{Author Biography}: Mandatory inclusion
	\item \textbf{Degree Type}: Master of Technology (M. TECH) or equivalent
	\item \textbf{Enhanced Documentation}: Comprehensive literature review and methodology
\end{itemize}

The configuration for postgraduate thesis must include author biography:

\begin{verbatim}
	% Publications by authors
	%%%%%%%%%%%%%%%%%%%%%%%%%%%%%%%%%%%%%%%%%%%%%%%%%%%%%%%%%%%%%%%%%%%%%%%
% Title: Publications by the Authors
% Purpose: List of publications by thesis authors
% Compiler: pdflatex
% OS: Manjaro 
% Version: V1
% Written on: July 05, 2025
% Revision Date: July 05, 2025
% Author: Kingsuk Majumdar
% Copyright (c) 2025 Kingsuk Majumdar
%%%%%%%%%%%%%%%%%%%%%%%%%%%%%%%%%%%%%%%%%%%%%%%%%%%%%%%%%%%%%%%%%%%%%%%

\chapter*{Publications by the Authors}
\addcontentsline{toc}{chapter}{Publications by the Authors}
\chaptermark{Publications by the Authors}
\thispagestyle{plain}

\section*{Journal Publications}
\begin{enumerate}
\item \textbf{Mitter, Pradosh Chandra}, Mitra, Tapesh Ranjan, and Ganguly, Lalmohan, ``IoT-Based Smart Grid Monitoring System with Machine Learning Integration,'' \textit{International Journal of Smart Grid and Clean Energy}, vol. 14, no. 2, pp. 45-58, 2025. DOI: 10.12720/sgce.14.2.45-58

\item Mitra, Tapesh Ranjan, \textbf{Mitter, Pradosh Chandra}, and Ganguly, Lalmohan, ``Machine Learning Algorithms for Power System Anomaly Detection: A Comparative Study,'' \textit{IEEE Access}, vol. 13, pp. 15234-15247, 2025. DOI: 10.1109/ACCESS.2025.3456789

\item \textbf{Ganguly, Lalmohan}, Mitter, Pradosh Chandra, and Mitra, Tapesh Ranjan, ``Wireless Sensor Networks for Smart Grid Applications: A Comprehensive Review,'' \textit{Renewable and Sustainable Energy Reviews}, vol. 145, article 111098, 2025. DOI: 10.1016/j.rser.2025.111098
\end{enumerate}

\section*{Conference Publications}
\begin{enumerate}
\item \textbf{Ganguly, Lalmohan}, Mitter, Pradosh Chandra, and Mitra, Tapesh Ranjan, ``Development of Wireless Sensor Network for Real-time Grid Monitoring,'' in \textit{Proceedings of IEEE International Conference on Power Electronics and Drives}, New Delhi, India, March 2025, pp. 234-239. DOI: 10.1109/IPED.2025.9123456

\item Mitter, Pradosh Chandra, \textbf{Mitra, Tapesh Ranjan}, and Ganguly, Lalmohan, ``Performance Evaluation of ML Algorithms in Smart Grid Applications,'' in \textit{National Conference on Advances in Electrical Engineering}, Dr. B. C. Roy Engineering College, Durgapur, February 2025, pp. 67-72.

\item \textbf{Mitra, Tapesh Ranjan}, Ganguly, Lalmohan, and Mitter, Pradosh Chandra, ``IoT Security Framework for Smart Grid Systems,'' in \textit{International Conference on Cybersecurity in Smart Grid}, IIT Kharagpur, January 2025, pp. 156-161.
\end{enumerate}

\section*{Under Review}
\begin{enumerate}
\item \textbf{Mitter, Pradosh Chandra}, Mitra, Tapesh Ranjan, Ganguly, Lalmohan, and Majumdar, Kingsuk, ``Comprehensive Analysis of IoT Security in Smart Grid Systems,'' \textit{Renewable and Sustainable Energy Reviews}, Elsevier. [Under Review - Submitted December 2024]

\item Ganguly, Lalmohan, \textbf{Mitter, Pradosh Chandra}, and Mitra, Tapesh Ranjan, ``Edge Computing for Real-time Smart Grid Data Processing,'' \textit{IEEE Transactions on Smart Grid}, IEEE. [Under Review - Submitted January 2025]
\end{enumerate}

\cleardoublepage
	
	% About the authors - REQUIRED FOR PG
	\include{Backmatter/AuthorBio} % Applicable for PG/PhD ONLY
\end{verbatim}

\section{Compilation Methods}
\label{sec:compilation}

\subsection{Offline Compilation in Manjaro Linux}
\label{subsec:offline_compilation}

For offline compilation in Manjaro Linux environment, the following procedure should be followed:

\subsubsection{Prerequisites Installation}
\label{subsubsec:prerequisites}

\begin{verbatim}
	# Update system repositories
	sudo pacman -Syu
	
	# Install complete LaTeX distribution
	sudo pacman -S texlive-most texlive-bibtexextra
	
	# Alternative: Install full TeX Live distribution
	sudo pacman -S texlive-core texlive-bin texlive-latexextra texlive-fontsextra
\end{verbatim}

\subsubsection{Compilation Process}
\label{subsubsec:compilation_process}

Navigate to the thesis template directory and execute the following commands:

\begin{verbatim}
	# Navigate to project directory
	cd /path/to/ug-thesis-template/
	
	# Create output directory
	mkdir -p OUTPUT
	
	# Primary compilation sequence
	pdflatex main.tex
	bibtex main
	pdflatex main.tex
	pdflatex main.tex
	
	# Move generated files to OUTPUT directory
	mv main.pdf OUTPUT/
	mv *.aux *.bbl *.blg *.log *.toc *.lof *.lot OUTPUT/ 2>/dev/null || true
\end{verbatim}

\subsubsection{Advanced Compilation Options}
\label{subsubsec:advanced_compilation}

For debugging and optimization:

\begin{verbatim}
	# Compilation with detailed logging
	pdflatex -interaction=nonstopmode -file-line-error main.tex > 
	compilation.log 2>&1
	
	# Draft mode compilation (faster for testing)
	pdflatex "\def\isdraft{1}%%%%%%%%%%%%%%%%%%%%%%%%%%%%%%%%%%%%%%%%%%%%%%%%%%%%%%%%%%%%%%%%%%%%%%%
% Title: UG/PG Thesis Main File - Clean Structure
% Purpose: Main file with user inputs only
% Compiler: pdflatex
% OS: Manjaro 
% Version: V3 (Reorganized)
% Written on: July 01, 2025
% Revision Date: July 06, 2025
% Author: Kingsuk Majumdar
% Copyright (c) 2025 Kingsuk Majumdar
%%%%%%%%%%%%%%%%%%%%%%%%%%%%%%%%%%%%%%%%%%%%%%%%%%%%%%%%%%%%%%%%%%%%%%%

\documentclass{thesis}

%%%%%%%%%% USER INPUT SECTION - MODIFY THIS SECTION ONLY %%%%%%%%%%

%% Thesis Information
\ThesisTitle{Long Thesis Title}
\ShortTitle{Short Thesis Title}
\Department{Department of Electrical Engineering}
\College{Dr. B. C. Roy Engineering College}
\University{Maulana Abul Kalam Azad University of Technology, West Bengal}
\DegreeType{Bachelor of Technology (B. TECH)}
\ThesisYear{2025}
\ThesisMonth{May}
\Location{Durgapur}
\AY{2024-2025}
\Address{Durgapur – 713206, West Bengal, India}
\Affiliation{(An Autonomous Institute, Affiliated To MAKAUT)}

%% Project Information
\GroupNo{Group 00}
\PaperName{Final Year Project Stage-II}
\PaperCode{PWEE881}

%% Number of Students (1-3)
\NumberOfStudents{3}

%% Student Information
\StudentOne{Pradosh Chandra Mitter}
\RollOne{18/EE/045}
\RegOne{184410301045}
\EmailOne{pradosh.mitter@bcrec.ac.in}
\PhotoOne{Figures/StudentOne_photo.jpg}

\StudentTwo{Tapesh Ranjan Mitter}
\RollTwo{18/EE/052}
\RegTwo{184410301052}
\EmailTwo{tapesh.mitter@bcrec.ac.in}
\PhotoTwo{Figures/StudentTwo_photo.jpg}

\StudentThree{Lalmohan Gonguly}
\RollThree{18/EE/063}
\RegThree{184410301063}
\EmailThree{lalmohan.gonguly@bcrec.ac.in}
\PhotoThree{Figures/StudentThree_photo.jpg}



%% Supervisor Information
\HasCoSupervisor{2} % 1=supervisor only, 2=both supervisor and co-supervisor
\Supervisor{Professor C.V. Raman}
\SupervisorDesignation{Professor}
\SupervisorEmail{cv.raman@bcrec.ac.in}
\SupervisorDept{Department of Electrical Engineering}

\CoSupervisor{Acharya Prafulla Chandra Ray}
\CoSupervisorDesignation{Assistant Professor}
\CoSupervisorEmail{pc.ray@bcrec.ac.in}
\CoSupervisorDept{Department of Electrical Engineering}

%% Head of Department
\HoD{Prof. Srinivasa Ramanujan}
\HoDDesignation{Professor \& Head}
\HoDDept{Department of Electrical Engineering}

%%%%%%%%%% END OF USER INPUT SECTION %%%%%%%%%%

\begin{document}

% Set line spacing
\onehalfspacing

%%%%%%%%%% FRONT MATTER %%%%%%%%%%
\frontmatter

% Title page
\makefrontcover

% Copyright page
\makecopyrightpage

% Declaration and Certificate
%%%%%%%%%%%%%%%%%%%%%%%%%%%%%%%%%%%%%%%%%%%%%%%%%%%%%%%%%%%%%%%%%%%%%%%
% Title: Declaration Page
% Purpose: Declaration page for UG thesis
% Compiler: pdflatex
% OS: Manjaro 
% Version: V3
% Written on: July 05, 2025
% Revision Date: July 05, 2025
% Author: Kingsuk Majumdar
% Copyright (c) 2025 Kingsuk Majumdar
%%%%%%%%%%%%%%%%%%%%%%%%%%%%%%%%%%%%%%%%%%%%%%%%%%%%%%%%%%%%%%%%%%%%%%%

% Put this file in Frontmatter/ directory as Declaration.tex

\thispagestyle{empty}
\vspace*{-0.5cm}

% Header with logo and college info
\begin{table}[H]
\centering
\begin{tabular}{@{}m{3cm}@{\hspace{1cm}}m{12cm}@{}}
%\IfFileExists{college_logo.png}{\includegraphics[width=2.5cm,height=2.5cm]{college_logo.png}}{\IfFileExists{college_logo.jpg}{\includegraphics[width=2.5cm,height=2.5cm]{college_logo.jpg}}{\framebox[2.5cm][c]{LOGO}}}
\includegraphics[width=0.95\linewidth]{college_logo.png} &
\begin{tabular}{@{}c@{}}  
\Large{\textbf{\MakeUppercase{\GetCollege}}}\\ 
\small{\GetAffiliation}\\
\textbf{\GetDepartment}\\
\textbf{\GetAddress}
\end{tabular}
\end{tabular}
\end{table}

\vspace{-1.0cm}
\noindent\rule{\textwidth}{2pt}
\vspace{-1.4cm}

\begin{center}
\textbf{\large \MakeUppercase{Declaration}}
\end{center}

\vspace{0.1cm}

We, the undersigned, hereby declare that the work presented in this project report entitled 
``\textbf{\GetThesisTitle}'' is the result of our own investigation carried out under the supervision of 
\textbf{\GetSupervisor}%
\ifnum\GetHasCoSupervisor>1\relax\ and \textbf{\GetCoSupervisor}\fi\ 
at \GetCollege, \GetLocation.

\vspace{0.1cm}

We further declare that this work has not been submitted to any other university or institution 
for the award of any degree or diploma. All the references used have been duly acknowledged.

\vspace{0.1cm}

We certify that the work embodied in this project work has been done by us and that all material 
from other sources have been properly and fully acknowledged.

\vspace{0.1cm}

\begin{center}
\GenerateSignatureList
\end{center}

\cleardoublepage
%%%%%%%%%%%%%%%%%%%%%%%%%%%%%%%%%%%%%%%%%%%%%%%%%%%%%%%%%%%%%%%%%%%%%%%
% Title: Certificate Page
% Purpose: Certificate page for UG thesis
% Compiler: pdflatex
% OS: Manjaro 
% Version: V5
% Written on: July 05, 2025
% Revision Date: July 05, 2025
% Author: Kingsuk Majumdar
% Copyright (c) 2025 Kingsuk Majumdar
%%%%%%%%%%%%%%%%%%%%%%%%%%%%%%%%%%%%%%%%%%%%%%%%%%%%%%%%%%%%%%%%%%%%%%%
% Setup background logo for certificate page only
\clearpage

% Put this file in Frontmatter/ directory as Certificate.tex

\thispagestyle{empty}
\vspace*{-0.5cm}

% Header with logo and college info
\begin{table}[H]
\centering
\begin{tabular}{@{}m{3cm}@{\hspace{1cm}}m{12cm}@{}}
%\IfFileExists{college_logo.png}{\includegraphics[width=2.5cm,height=2.5cm]{college_logo.png}}{\IfFileExists{college_logo.jpg}{\includegraphics[width=2.5cm,height=2.5cm]{college_logo.jpg}}{\framebox[2.5cm][c]{LOGO}}} 
\includegraphics[width=0.95\linewidth]{college_logo.png}&
\begin{tabular}{@{}c@{}}  
\Large{\textbf{\MakeUppercase{\GetCollege}}}\\ 
\small{\GetAffiliation}\\
\textbf{\GetDepartment}\\
\textbf{\GetAddress}
\end{tabular}
\end{tabular}
\end{table}

\vspace{-1.0cm}
\noindent\rule{\textwidth}{2pt}
\vspace{-0.8cm}
\begin{center}
\textbf{\large \MakeUppercase{Certificate of Approval}}
\end{center}

\vspace{0.1cm}
This report is hereby approved as a creditable work for final year project 
[\GetPaperName\ (\GetPaperCode)] on ``\textbf{\GetThesisTitle}'' carried out and
presented by \textbf{\GetGroupNo}:

\vspace{-0.3cm}
\begin{center}
\GenerateStudentList
\end{center}

\vspace{-0.3cm}
in partial fulfillment of the requirements for the award of Degree of \GetDegreeType\ in \GetDepartment\ from \GetCollege,
\GetLocation\ under the supervision of \textbf{\GetSupervisor}%
\ifnum\GetHasCoSupervisor>1\relax\ and \textbf{\GetCoSupervisor}\fi\ as per the requirement of the
curriculum of \GetUniversity\ during the academic year \GetAY.

\vspace{0.2cm}

% Co-Supervisor section (if applicable) - spans full width
\ifnum\GetHasCoSupervisor>1\relax
\begin{flushleft}
(Signature of Co-Supervisor)\\[0.1cm]
\textbf{\GetCoSupervisor}\\
\GetCoSupervisorDesignation\\
\GetCoSupervisorDept\\
\GetCollege\\[0.1cm]
Place: \underline{\hspace{3cm}} \hspace{2cm} Date: \underline{\hspace{3cm}}
\end{flushleft}
\vspace{0.3cm}
\fi

% Supervisor and HoD section - side by side layout
\begin{minipage}[t]{0.47\textwidth}
\begin{flushleft}
(Signature of Supervisor)\\[0.2cm]
\textbf{\GetSupervisor}\\
\GetSupervisorDesignation\\
\GetSupervisorDept\\
\GetCollege\\[0.1cm]
Place: \underline{\hspace{2.5cm}}\\
Date: \underline{\hspace{2.5cm}}
\end{flushleft}
\end{minipage}%
\hfill
\begin{minipage}[t]{0.47\textwidth}
\begin{flushright}
(Signature of Head of Department)\\[0.2cm]
\textbf{\GetHoD}\\
\GetHoDDesignation\\
\GetHoDDept\\
\GetCollege\\[0.1cm]
Place: \underline{\hspace{2.5cm}}\\
Date: \underline{\hspace{2.5cm}}
\end{flushright}
\end{minipage}
\clearpage

\cleardoublepage


% Acknowledgment and Abstract
%%%%%%%%%%%%%%%%%%%%%%%%%%%%%%%%%%%%%%%%%%%%%%%%%%%%%%%%%%%%%%%%%%%%%%%
% Title: Acknowledgment Page
% Purpose: Acknowledgment page for UG thesis
% Compiler: pdflatex
% OS: Manjaro 
% Version: V2
% Written on: July 05, 2025
% Revision Date: July 05, 2025
% Author: Kingsuk Majumdar
% Copyright (c) 2025 Kingsuk Majumdar
%%%%%%%%%%%%%%%%%%%%%%%%%%%%%%%%%%%%%%%%%%%%%%%%%%%%%%%%%%%%%%%%%%%%%%%

% Put this file in Frontmatter/Acknowledgment.tex

\chapter*{Acknowledgment}
\addcontentsline{toc}{chapter}{Acknowledgment}
\chaptermark{Acknowledgment}
\thispagestyle{plain}
\justifying 
We express our sincere gratitude to our project supervisor \textbf{\GetSupervisor}%
\ifnum\GetHasCoSupervisor>1\relax\ and co-supervisor \textbf{\GetCoSupervisor}\fi\ for their constant guidance, support, and encouragement throughout this project work. Their valuable suggestions and constructive criticism helped us in completing this project successfully.

We are thankful to \textbf{\GetHoD}, \GetHoDDesignation\ of \GetHoDDept, for providing necessary facilities and resources for our project work.

We also extend our thanks to all faculty members and staff of the \GetDepartment\ for their support and cooperation during the course of this project.

We acknowledge the technical support provided by the laboratory staff and the library resources that were instrumental in our research work.

Finally, we are grateful to our family members and friends for their continuous support and motivation throughout this project.

\vspace{2cm}

\begin{flushright}
	\GetStudentOne\\
	Roll No: \GetRollOne\\[1cm]
	
	\ifnum\GetNumberOfStudents>1\relax
	\GetStudentTwo\\
	Roll No: \GetRollTwo\\[1cm]
	\fi
	
	\ifnum\GetNumberOfStudents>2\relax
	\GetStudentThree\\
	Roll No: \GetRollThree\\[1cm]
	\fi
	
	\ifnum\GetNumberOfStudents>3\relax
	\GetStudentFour\\
	Roll No: \GetRollFour\\[1cm]
	\fi
	
	\ifnum\GetNumberOfStudents>4\relax
	\GetStudentFive\\
	Roll No: \GetRollFive\\
	\fi
\end{flushright}

\cleardoublepage
%%%%%%%%%%%%%%%%%%%%%%%%%%%%%%%%%%%%%%%%%%%%%%%%%%%%%%%%%%%%%%%%%%%%%%%
% Title: Abstract Page
% Purpose: Abstract for UG thesis with improved formatting
% Compiler: pdflatex
% OS: Manjaro 
% Version: V2
% Written on: July 05, 2025
% Revision Date: July 05, 2025
% Author: Kingsuk Majumdar
% Copyright (c) 2025 Kingsuk Majumdar
%%%%%%%%%%%%%%%%%%%%%%%%%%%%%%%%%%%%%%%%%%%%%%%%%%%%%%%%%%%%%%%%%%%%%%%

% Put this file in Frontmatter/Abstract.tex

\chapter*{Abstract}
\addcontentsline{toc}{chapter}{Abstract}
\chaptermark{Abstract}
\thispagestyle{plain}

With the rapid advancement in technology and increasing demand for efficient power management, smart grid systems have become essential for modern electrical infrastructure. This project presents the development of a comprehensive smart grid monitoring system that integrates Internet of Things (IoT) technology with machine learning algorithms to enhance grid reliability, efficiency, and sustainability.

The proposed system utilizes various sensors and IoT devices to collect real-time data from different components of the electrical grid including voltage, current, frequency, and power quality parameters. The collected data is transmitted through wireless communication protocols to a central monitoring station where machine learning algorithms process and analyze the information to detect anomalies, predict failures, and optimize grid operations.

The machine learning component employs artificial neural networks and support vector machines to classify normal and abnormal grid conditions. The system also incorporates predictive maintenance capabilities using time-series analysis and regression techniques to forecast equipment failures and schedule maintenance activities proactively.

A user-friendly web-based interface has been developed to visualize real-time grid status, historical trends, and alert notifications. The system also includes automated control features that can respond to critical situations by adjusting load distribution or isolating faulty sections.

Simulation results demonstrate that the proposed system can effectively monitor grid parameters with 95\% accuracy in anomaly detection and reduce downtime by 30\% through predictive maintenance. The IoT-based architecture ensures scalability and cost-effectiveness, making it suitable for implementation in both urban and rural electrical networks.

The project contributes to the advancement of smart grid technology and provides a foundation for future research in intelligent power system monitoring and control. The developed system addresses the critical need for real-time monitoring and predictive maintenance in modern electrical grids, offering significant improvements in reliability and operational efficiency.

\cleardoublepage

% Table of Contents
\setcounter{tocdepth}{4}
\tableofcontents

% List of Figures
\listoffigures

% List of Tables
\listoftables

% List of Abbreviations
%%%%%%%%%%%%%%%%%%%%%%%%%%%%%%%%%%%%%%%%%%%%%%%%%%%%%%%%%%%%%%%%%%%%%%%
% Title: List of Abbreviations
% Purpose: Abbreviations list for UG thesis
% Compiler: pdflatex
% OS: Manjaro 
% Version: V1
% Written on: July 05, 2025
% Revision Date: July 05, 2025
% Author: Kingsuk Majumdar
% Copyright (c) 2025 Kingsuk Majumdar
%%%%%%%%%%%%%%%%%%%%%%%%%%%%%%%%%%%%%%%%%%%%%%%%%%%%%%%%%%%%%%%%%%%%%%%

% Put this file in Frontmatter/Acronyms.tex

\chapter*{List of Abbreviations}
\addcontentsline{toc}{chapter}{List of Abbreviations}
\chaptermark{List of Abbreviations}
\thispagestyle{plain}

\begin{abbrv}
	\item[AI] Artificial Intelligence
	\item[ANN] Artificial Neural Network
	\item[API] Application Programming Interface
	\item[CNN] Convolutional Neural Network
	\item[DER] Distributed Energy Resources
	\item[DSO] Distribution System Operator
	\item[GUI] Graphical User Interface
	\item[HMI] Human Machine Interface
	\item[HTTP] Hypertext Transfer Protocol
	\item[IoT] Internet of Things
	\item[JSON] JavaScript Object Notation
	\item[KNN] k-Nearest Neighbors
	\item[ML] Machine Learning
	\item[MQTT] Message Queuing Telemetry Transport
	\item[PMU] Phasor Measurement Unit
	\item[RF] Random Forest
	\item[SCADA] Supervisory Control and Data Acquisition
	\item[SVM] Support Vector Machine
	\item[TCP] Transmission Control Protocol
	\item[THD] Total Harmonic Distortion
	\item[UI] User Interface
	\item[WiFi] Wireless Fidelity
	\item[WSN] Wireless Sensor Network
\end{abbrv}

\cleardoublepage

%%%%%%%%%% MAIN CONTENT %%%%%%%%%%
\mainmatter

% Setup headers for main content
\setupheaders

% Include individual chapters
%%%%%%%%%%%%%%%%%%%%%%%%%%%%%%%%%%%%%%%%%%%%%%%%%%%%%%%%%%%%%%%%%%%%%%%
% Title: Chapter 1 - Introduction
% Purpose: Introduction chapter for smart grid monitoring system thesis
% Compiler: pdflatex
% OS: Manjaro 
% Version: V1
% Written on: July 05, 2025
% Revision Date: July 05, 2025
% Author: Kingsuk Majumdar
% Copyright (c) 2025 Kingsuk Majumdar
%%%%%%%%%%%%%%%%%%%%%%%%%%%%%%%%%%%%%%%%%%%%%%%%%%%%%%%%%%%%%%%%%%%%%%%

\chapter{Introduction}
\label{chap:introduction}
\justifying
\section{Background and Motivation}
\label{sec:background}

The electrical power grid forms the backbone of modern society, supplying energy to residential, commercial, and industrial consumers. Traditional power grids were designed as centralized systems with unidirectional power flow from large generation facilities to end consumers. However, the increasing integration of renewable energy sources, distributed generation, and evolving consumer demands have necessitated the transformation of conventional grids into intelligent, bidirectional smart grids.

Smart grids represent a paradigm shift in power system operation, incorporating advanced communication technologies, real-time monitoring capabilities, and automated control systems. The integration of Internet of Things (IoT) devices and machine learning algorithms has opened new avenues for enhancing grid reliability, efficiency, and sustainability. These technologies enable real-time data collection, predictive analytics, and autonomous decision-making, which are essential for managing the complexity of modern power systems.

The motivation for this research stems from the critical need to address the challenges faced by conventional grid monitoring systems. Traditional monitoring approaches suffer from limited real-time visibility, manual fault detection processes, and reactive maintenance strategies. These limitations result in increased downtime, higher operational costs, and reduced system reliability. The development of an intelligent monitoring system that combines IoT sensors with machine learning algorithms can significantly improve grid performance and operational efficiency.

\section{Problem Statement}
\label{sec:problem_statement}

The main challenges addressed in this research work are:

\subsection{Limited Real-time Monitoring}
Conventional grid monitoring systems rely on periodic manual inspections and limited sensor coverage, resulting in delayed detection of anomalies and faults. This lack of real-time visibility hampers the ability to respond quickly to system disturbances and optimize grid operations.

\subsection{Reactive Maintenance Approach}
Traditional maintenance strategies are primarily reactive, addressing problems only after they occur. This approach leads to unexpected equipment failures, increased downtime, and higher maintenance costs. There is a critical need for predictive maintenance capabilities that can forecast potential failures and enable proactive intervention.

\subsection{Inadequate Data Analytics}
Existing monitoring systems generate vast amounts of data but lack sophisticated analytics capabilities to extract meaningful insights. The absence of intelligent data processing and pattern recognition limits the ability to identify trends, predict anomalies, and optimize system performance.

\subsection{Poor System Integration}
Many legacy monitoring systems operate in isolation without proper integration capabilities. This fragmented approach hinders comprehensive system analysis and coordinated control actions. There is a need for integrated monitoring solutions that can provide holistic system visibility and coordinated response mechanisms.

\section{Research Objectives}
\label{sec:objectives}

The primary objectives of this research work are:

\begin{enumerate}
\item \textbf{Design and Development of IoT-based Data Acquisition System:} To develop a comprehensive sensor network using IoT devices for real-time collection of critical grid parameters including voltage, current, frequency, power quality, and environmental conditions.

\item \textbf{Implementation of Machine Learning Algorithms:} To implement and evaluate various machine learning algorithms for anomaly detection, pattern recognition, and predictive maintenance in smart grid applications.

\item \textbf{Development of Intelligent Monitoring Platform:} To create an integrated monitoring platform that combines real-time data visualization, automated alerting, and decision support capabilities.

\item \textbf{Performance Evaluation and Validation:} To conduct comprehensive testing and validation of the developed system through simulation studies and prototype implementation.

\item \textbf{Development of Predictive Maintenance Framework:} To establish a predictive maintenance framework that can forecast equipment failures and optimize maintenance schedules.
\end{enumerate}

\section{Scope and Limitations}
\label{sec:scope_limitations}

\subsection{Scope of the Work}
This research focuses on the development of a smart grid monitoring system with the following scope:

\begin{itemize}
\item Development of IoT sensor networks for distribution-level monitoring
\item Implementation of machine learning algorithms for anomaly detection and predictive analytics
\item Design of web-based monitoring interface and visualization tools
\item Integration of real-time data processing and automated alerting systems
\item Performance evaluation through simulation and prototype testing
\end{itemize}

\subsection{Limitations}
The limitations of this study include:

\begin{itemize}
\item The prototype implementation is limited to laboratory-scale testing and simulation
\item The study focuses primarily on distribution-level monitoring and does not cover transmission-level applications
\item Cybersecurity aspects are considered but not extensively implemented in the prototype
\item The economic analysis is limited to conceptual cost-benefit evaluation
\item Field testing is not performed due to resource and time constraints
\end{itemize}

\section{Research Methodology}
\label{sec:methodology_overview}

The research methodology adopted in this work follows a systematic approach consisting of the following phases:

\subsection{Literature Review and Technology Analysis}
A comprehensive review of existing smart grid monitoring technologies, IoT applications in power systems, and machine learning techniques for grid analytics is conducted to identify research gaps and establish the theoretical foundation.

\subsection{System Design and Architecture Development}
Based on the literature review and identified requirements, a detailed system architecture is designed incorporating IoT sensors, communication protocols, data processing algorithms, and user interface components.

\subsection{Hardware and Software Development}
The system implementation involves:
\begin{itemize}
\item Selection and integration of appropriate IoT sensors and communication modules
\item Development of data acquisition and processing software
\item Implementation of machine learning algorithms for anomaly detection and prediction
\item Design and development of web-based monitoring interface
\end{itemize}

\subsection{Testing and Validation}
Comprehensive testing is performed through:
\begin{itemize}
\item Laboratory testing of individual components and integrated system
\item Simulation studies using real-world grid data
\item Performance evaluation and comparison with existing methods
\item Validation of machine learning model accuracy and reliability
\end{itemize}

\section{Contributions and Novelty}
\label{sec:contributions}

The main contributions of this research work include:

\begin{enumerate}
\item \textbf{Integrated IoT-ML Framework:} Development of a novel framework that seamlessly integrates IoT sensors with machine learning algorithms for comprehensive grid monitoring and analytics.

\item \textbf{Multi-parameter Anomaly Detection:} Implementation of advanced machine learning techniques for simultaneous monitoring and analysis of multiple grid parameters to detect various types of anomalies and disturbances.

\item \textbf{Predictive Maintenance System:} Development of a predictive maintenance framework using time-series analysis and machine learning to forecast equipment failures and optimize maintenance schedules.

\item \textbf{Scalable Architecture:} Design of a scalable and modular system architecture that can be easily extended and adapted for different grid configurations and requirements.

\item \textbf{Real-time Visualization Platform:} Creation of an intuitive web-based interface for real-time monitoring, historical analysis, and interactive system control.
\end{enumerate}

\section{Thesis Organization}
\label{sec:thesis_organization}

This thesis is organized into six chapters, each addressing specific aspects of the research work:

\textbf{Chapter 1: Introduction} provides the background, motivation, problem statement, objectives, scope, and overview of the research methodology. It establishes the foundation and context for the entire research work.

\textbf{Chapter 2: Literature Review} presents a comprehensive review of existing literature on smart grid technologies, IoT applications in power systems, machine learning techniques for grid monitoring, and related research work. This chapter identifies research gaps and positions the current work within the broader research landscape.

\textbf{Chapter 3: Methodology} describes the detailed research methodology, system architecture design, hardware and software requirements, and implementation approach. It provides the technical foundation for the system development.

\textbf{Chapter 4: Implementation and Design} presents the detailed implementation of the smart grid monitoring system, including hardware integration, software development, machine learning algorithm implementation, and user interface design.

\textbf{Chapter 5: Results and Analysis} provides comprehensive results from testing and validation studies, performance evaluation metrics, comparison with existing methods, and discussion of findings. This chapter demonstrates the effectiveness and capabilities of the developed system.

\textbf{Chapter 6: Conclusion and Future Work} summarizes the research findings, highlights the main contributions, discusses limitations, and suggests directions for future research and development.

The thesis also includes appendices containing detailed circuit diagrams, source code, test results, and component specifications that support the main research work.

\section{Chapter Summary}
\label{sec:chapter1_summary}

This chapter has established the foundation for the research work on developing a smart grid monitoring system using IoT and machine learning technologies. The background and motivation for the research have been presented, highlighting the critical need for intelligent monitoring solutions in modern power systems. The problem statement clearly identifies the limitations of existing monitoring approaches and the challenges that need to be addressed.

The research objectives have been defined to provide a roadmap for the development of an integrated IoT-ML framework for smart grid monitoring. The scope and limitations of the work have been outlined to set appropriate expectations and boundaries for the research. The research methodology provides a systematic approach for achieving the defined objectives through literature review, system design, implementation, and validation phases.

The main contributions and novelty of the research have been highlighted, emphasizing the integrated approach and advanced capabilities of the proposed system. Finally, the thesis organization provides a clear structure for presenting the research work and findings in subsequent chapters.

The next chapter will present a comprehensive literature review of existing technologies and research work related to smart grid monitoring, IoT applications, and machine learning techniques, establishing the theoretical foundation for the proposed system.

\chapter{Literature Review}

\section{Introduction}
This chapter presents a comprehensive review of existing literature related to the development of smart grid monitoring systems. The review is divided into thematic sections covering smart grid architectures, IoT integration, and machine learning applications. The aim is to understand the evolution, current advancements, and limitations of prior work to justify the scope of this research.

\section{Smart Grid and Monitoring Systems}
Smart grids represent the modernization of traditional electrical grids by incorporating advanced communication and automation technologies. Various studies have addressed the implementation challenges and advantages of smart grids in power systems \cite{ieee2030,fang2012smart}. Effective monitoring systems are essential for fault detection, real-time decision-making, and load management.

\section{Integration of IoT in Smart Grids}
The Internet of Things (IoT) enables connectivity between sensors, meters, and control systems, facilitating real-time data acquisition and remote monitoring \cite{zanella2014internet}. Several researchers have explored low-cost IoT-based architectures for distributed grid monitoring \cite{ghosh2017iot}, emphasizing the use of microcontrollers, wireless protocols, and cloud platforms.

\section{Machine Learning Applications}
Machine learning (ML) offers predictive and adaptive capabilities in grid analysis. Techniques such as support vector machines, decision trees, and neural networks have been used for load forecasting, fault classification, and energy consumption optimization \cite{mohamed2019machine, singh2020review}. The integration of ML with IoT enhances the intelligence and automation of smart grids.

\section{Comparative Analysis of Related Work}
Table~\ref{tab:relatedwork} summarizes key contributions in literature, comparing methods, tools used, and performance metrics.

\begin{table}[H]
\centering
\caption{Comparison of Selected Literature on Smart Grid Monitoring}
\label{tab:relatedwork}
\begin{tabular}{|p{3.5cm}|p{3cm}|p{3cm}|p{3cm}|}
\hline
\textbf{Author(s)} & \textbf{Technology Used} & \textbf{Focus Area} & \textbf{Remarks} \\
\hline
Fang et al. (2012) \cite{fang2012smart} & Communication and Security & Smart Grid Framework & Early overview of challenges and architecture \\
\hline
Zanella et al. (2014) \cite{zanella2014internet} & IoT Architecture & Urban IoT for Smart Cities & Demonstrated scalability and cost-effectiveness \\
\hline
Mohamed et al. (2019) \cite{mohamed2019machine} & ML Algorithms & Load Forecasting & High accuracy using ANN \\
\hline
Singh et al. (2020) \cite{singh2020review} & Hybrid ML Models & Fault Detection & Emphasized real-time learning models \\
\hline
\end{tabular}
\end{table}

\section{Research Gaps Identified}
From the literature, several gaps have been identified:
\begin{itemize}
    \item Lack of integrated systems combining both IoT and ML for comprehensive monitoring.
    \item Limited real-time deployment and testing on live grid systems.
    \item Data privacy and security remain less addressed in existing IoT-based models.
\end{itemize}

\section{Summary}
The literature demonstrates promising advancements in smart grid monitoring through IoT and ML. However, challenges in scalability, real-time performance, and integration offer significant scope for this research. The next chapter will elaborate on the methodology adopted in this work.


%%%%%%%%%%%%%%%%%%%%%%%%%%%%%%%%%%%%%%%%%%%%%%%%%%%%%%%%%%%%%%%%%%%%%%%
% Title: Chapter 2 - Tables and Data Presentation
% Purpose: Simple IEEE standard tables (3 types only)
% Compiler: pdflatex
% OS: Manjaro 
% Version: V5 (Simplified)
% Written on: July 06, 2025
% Revision Date: July 06, 2025
% Author: Kingsuk Majumdar
% Copyright (c) 2025 Kingsuk Majumdar
%%%%%%%%%%%%%%%%%%%%%%%%%%%%%%%%%%%%%%%%%%%%%%%%%%%%%%%%%%%%%%%%%%%%%%%

\chapter{Tables and Data Presentation}
\label{chap:tables}

This chapter demonstrates three essential table types following IEEE standards for technical documentation: simple tables, long tables with page breaks, and landscape tables.

\section{Simple Table}
\label{sec:simple_table}

Table \ref{tab:system_parameters} presents the basic system parameters for smart grid monitoring implementation \cite{kundur1994power}. This demonstrates the standard IEEE table format with proper caption placement and referencing.

\begin{table}[htbp]
	\centering
	\caption{Smart Grid System Parameters}
	\label{tab:system_parameters}
	\begin{tabular}{|l|c|c|c|}
		\hline
		\textbf{Parameter} & \textbf{Symbol} & \textbf{Value} & \textbf{Unit} \\
		\hline
		Nominal Voltage & $V_n$ & 11.0 & kV \\
		System Frequency & $f$ & 50 & Hz \\
		Power Factor & $\cos\phi$ & 0.85 & -- \\
		Transformer Rating & $S_T$ & 5.0 & MVA \\
		Line Resistance & $R$ & 0.125 & $\Omega$/km \\
		Line Reactance & $X$ & 0.345 & $\Omega$/km \\
		Communication Range & -- & 500 & m \\
		Operating Temperature & $T_{op}$ & -10 to +65 & $^{\circ}$C \\
		\hline
	\end{tabular}
\end{table}

\section{Long Table with Page Breaks}
\label{sec:long_table}

Table \ref{tab:equipment_list} demonstrates the longtable environment for tables that span multiple pages \cite{farhangi2010smart}. The table automatically handles page breaks while maintaining consistent headers and formatting.

\begin{longtable}{|c|l|c|l|}
	\caption{Equipment Inventory for Smart Grid Implementation} 
	\label{tab:equipment_list} \\
	\hline
	\textbf{ID} & \textbf{Equipment Name} & \textbf{Qty} & \textbf{Status} \\
	\hline
	\endfirsthead
	
	\multicolumn{4}{c}%
	{{\bfseries Table \thetable\ Continued from previous page}} \\
	\hline
	\textbf{ID} & \textbf{Equipment Name} & \textbf{Qty} & \textbf{Status} \\
	\hline
	\endhead
	
	\hline \multicolumn{4}{|r|}{{Continued on next page}} \\ \hline
	\endfoot
	
	\hline
	\endlastfoot
	
	\multicolumn{4}{|c|}{\textbf{Power Transformers}} \\
	\hline
	PT001 & Distribution Transformer 1000 kVA & 5 & Installed \\
	PT002 & Step-up Transformer 5 MVA & 2 & Installed \\
	PT003 & Isolation Transformer 500 kVA & 8 & Ordered \\
	PT004 & Auto Transformer 2 MVA & 3 & Testing \\
	PT005 & Grounding Transformer 750 kVA & 4 & Installed \\
	\hline
	
	\multicolumn{4}{|c|}{\textbf{Protection Equipment}} \\
	\hline
	PD001 & Circuit Breaker SF6 33 kV & 12 & Installed \\
	PD002 & Vacuum Circuit Breaker 11 kV & 20 & Installed \\
	PD003 & Load Break Switch 33 kV & 15 & Testing \\
	PD004 & Disconnect Switch 11 kV & 25 & Installed \\
	PD005 & Surge Arresters 33 kV & 50 & Installed \\
	PD006 & Current Transformers 1000/5A & 60 & Installed \\
	PD007 & Voltage Transformers 33kV/110V & 45 & Installed \\
	PD008 & Digital Protection Relays & 30 & Testing \\
	\hline
	
	\multicolumn{4}{|c|}{\textbf{Smart Grid Components}} \\
	\hline
	SG001 & Phasor Measurement Units 50Hz & 8 & Installed \\
	SG002 & Smart Meters 230V & 500 & Installed \\
	SG003 & Data Concentrator 1000 nodes & 10 & Testing \\
	SG004 & Communication Gateway Ethernet & 15 & Installed \\
	SG005 & Weather Station Multi-sensor & 5 & Installed \\
	SG006 & SCADA Server High-end & 2 & Testing \\
	SG007 & HMI Workstation Industrial PC & 6 & Installed \\
	SG008 & Historian Database 10TB & 1 & Testing \\
	\hline
	
	\multicolumn{4}{|c|}{\textbf{Renewable Energy}} \\
	\hline
	RE001 & Solar PV Modules 250W & 200 & Installed \\
	RE002 & Solar Inverters 10kW & 25 & Installed \\
	RE003 & Wind Turbines 100kW & 3 & Testing \\
	RE004 & Battery Storage 500kWh & 5 & Ordered \\
	RE005 & MPPT Charge Controllers 60A & 30 & Installed \\
	RE006 & Monitoring System Complete & 1 & Testing \\
	\hline
	
	\multicolumn{4}{|c|}{\textbf{Communication Equipment}} \\
	\hline
	CE001 & Fiber Optic Cable Single-mode & 5000m & Installed \\
	CE002 & Ethernet Switches 24-port & 20 & Installed \\
	CE003 & Industrial Wireless Routers & 15 & Installed \\
	CE004 & RS485/Ethernet Converters & 50 & Installed \\
	CE005 & GPS Clock IEEE 1588 & 5 & Testing \\
	CE006 & Network Security Firewall & 8 & Testing \\
	\hline
	
	\multicolumn{4}{|c|}{\textbf{Control Systems}} \\
	\hline
	CS001 & Modular PLC Systems & 12 & Installed \\
	CS002 & RTU Protocol Converters & 8 & Testing \\
	CS003 & Variable Frequency Drives & 25 & Installed \\
	CS004 & Soft Starters 50HP & 15 & Installed \\
	CS005 & Power Factor Controller & 10 & Testing \\
	CS006 & Voltage Regulators 33kV & 8 & Ordered \\
	\hline
	
	\multicolumn{4}{|c|}{\textbf{Instrumentation}} \\
	\hline
	IN001 & Power Quality Meters Class A & 15 & Installed \\
	IN002 & Smart Energy Meters & 100 & Installed \\
	IN003 & RTD Temperature Sensors Pt100 & 80 & Installed \\
	IN004 & Pressure Transmitters 4-20mA & 20 & Installed \\
	IN005 & Ultrasonic Flow Meters & 12 & Testing \\
	IN006 & Capacitive Level Sensors & 25 & Installed \\
	IN007 & Wireless Vibration Monitors & 10 & Testing \\
	\hline
\end{longtable}
\vspace{-1cm}
\section{Landscape Table}
\label{sec:landscape_table}

Table \ref{tab:power_analysis} presents power flow analysis results in landscape orientation to accommodate wide data sets \cite{wood2013power}. The landscape environment allows for tables that require more horizontal space than standard portrait orientation permits.
\vspace{-3cm}
\begin{landscape}
	\begin{table}[htbp]
		\centering
		\caption{Power Flow Analysis Results for Different Operating Conditions}
		\label{tab:power_analysis}
		\begin{tabular}{|c|c|c|c|c|c|c|c|c|c|}
			\hline
			\multirow{2}{*}{\textbf{Bus}} & \multirow{2}{*}{\textbf{Type}} & \multicolumn{4}{c|}{\textbf{Light Load Condition (50\%)}} & \multicolumn{4}{c|}{\textbf{Heavy Load Condition (120\%)}} \\
			\cline{3-10}
			& & \textbf{V (pu)} & \textbf{Angle (deg)} & \textbf{P (MW)} & \textbf{Q (MVAr)} & \textbf{V (pu)} & \textbf{Angle (deg)} & \textbf{P (MW)} & \textbf{Q (MVAr)} \\
			\hline
			1 & Slack & 1.000 & 0.00 & 85.2 & 32.4 & 1.000 & 0.00 & 195.8 & 78.2 \\
			2 & PV & 1.050 & -2.15 & 45.0 & 18.5 & 1.020 & -5.42 & 108.0 & 44.8 \\
			3 & PQ & 1.035 & -3.28 & -20.0 & -8.5 & 0.985 & -8.67 & -48.0 & -20.4 \\
			4 & PQ & 1.028 & -4.12 & -15.0 & -6.2 & 0.978 & -9.85 & -36.0 & -14.9 \\
			5 & PV & 1.040 & -2.98 & 30.0 & 12.8 & 1.010 & -7.25 & 72.0 & 30.7 \\
			6 & PQ & 1.025 & -5.47 & -12.5 & -5.1 & 0.965 & -12.34 & -30.0 & -12.2 \\
			7 & PQ & 1.018 & -6.23 & -8.5 & -3.4 & 0.958 & -14.12 & -20.4 & -8.3 \\
			8 & PQ & 1.012 & -7.15 & -10.2 & -4.1 & 0.951 & -15.86 & -24.5 & -9.9 \\
			\hline
			\multicolumn{2}{|c|}{\textbf{Total Generation}} & \textbf{160.2} & -- & -- & \textbf{63.7} & \textbf{375.8} & -- & -- & \textbf{153.7} \\
			\multicolumn{2}{|c|}{\textbf{Total Load}} & \textbf{66.2} & -- & -- & \textbf{27.3} & \textbf{158.9} & -- & -- & \textbf{65.7} \\
			\multicolumn{2}{|c|}{\textbf{Total Losses}} & \textbf{94.0} & -- & -- & \textbf{36.4} & \textbf{216.9} & -- & -- & \textbf{88.0} \\
			\hline
		\end{tabular}
	\end{table}
\end{landscape}

\section{IEEE Table Standards and Citations}
\label{sec:ieee_standards}

All tables in this chapter follow IEEE formatting standards with:

\begin{itemize}
	\item Captions placed above tables using \texttt{\textbackslash caption\{\}} command
	\item Sequential numbering within chapters (Table 2.1, 2.2, 2.3)
	\item Proper citation format: ``Table \ref{tab:system_parameters} shows...''
	\item Consistent use of horizontal lines and column headers
	\item Units clearly specified in parentheses or separate columns
	\item Mathematical symbols properly formatted in math mode
\end{itemize}

The three table types demonstrated represent the most common requirements in electrical engineering documentation: basic parameter tables, equipment inventories requiring page breaks, and wide analytical results requiring landscape orientation.
%%%%%%%%%%%%%%%%%%%%%%%%%%%%%%%%%%%%%%%%%%%%%%%%%%%%%%%%%%%%%%%%%%%%%%%
% Title: Chapter 3 - Figures and Graphical Representation
% Purpose: Simple IEEE standard figures (4 types only)
% Compiler: pdflatex
% OS: Manjaro 
% Version: V4 (Simplified)
% Written on: July 06, 2025
% Revision Date: July 06, 2025
% Author: Kingsuk Majumdar
% Copyright (c) 2025 Kingsuk Majumdar
%%%%%%%%%%%%%%%%%%%%%%%%%%%%%%%%%%%%%%%%%%%%%%%%%%%%%%%%%%%%%%%%%%%%%%%

\chapter{Figures and Graphical Representation}
\label{chap:figures}

This chapter demonstrates four essential figure types following IEEE standards: resizable figures, 2x2 subfigure arrangements, figures within tables, and proper IEEE citation methods.

\section{Resizable Figure}
\label{sec:resizable_figure}

Figure \ref{fig:smart_grid_architecture} shows a smart grid system architecture that automatically adjusts to page width using adjustbox \cite{farhangi2010smart}.

\begin{figure}[htbp]
	\centering
	\adjustbox{width=0.8\textwidth,center}{%
		\IfFileExists{Figures/Chapter03/smart_grid.jpg}{%
			\includegraphics{Figures/Chapter03/smart_grid.jpg}
		}{%
			\IfFileExists{Figures/Chapter03/smart_grid.png}{%
				\includegraphics{Figures/Chapter03/smart_grid.png}
			}{%
				\fbox{\begin{minipage}{8cm}
						\centering
						\vspace{2cm}
						\textbf{Smart Grid Architecture}\\
						\vspace{0.5cm}
						Generation $\rightarrow$ Transmission $\rightarrow$ Distribution\\
						\vspace{0.5cm}
						[Image: smart\_grid.jpg/png not found]
						\vspace{2cm}
				\end{minipage}}
			}%
		}%
	}
	\caption{Smart grid system architecture showing integration of renewable energy sources and communication networks}
	\label{fig:smart_grid_architecture}
\end{figure}

\section{Subfigures in 2x2 Format}
\label{sec:subfigures}

Figure \ref{fig:control_systems} demonstrates the 2x2 subfigure arrangement showing different control system responses \cite{akagi2007instantaneous}.

\begin{figure}[htbp]
	\centering
	\subfigure[PI Controller Response]{
		\IfFileExists{Figures/Chapter03/pi_response.jpg}{%
			\includegraphics[width=0.45\textwidth]{Figures/Chapter03/pi_response.jpg}
		}{%
			\fbox{\begin{minipage}{0.4\textwidth}
					\centering
					\vspace{1cm}
					\textbf{PI Response}\\
					Overshoot: 15\%\\
					Settling: 2.5s
					\vspace{1cm}
			\end{minipage}}
		}
		\label{fig:pi_response}
	}
	\hfill
	\subfigure[PID Controller Response]{
		\IfFileExists{Figures/Chapter03/pid_response.jpg}{%
			\includegraphics[width=0.45\textwidth]{Figures/Chapter03/pid_response.jpg}
		}{%
			\fbox{\begin{minipage}{0.4\textwidth}
					\centering
					\vspace{1cm}
					\textbf{PID Response}\\
					Overshoot: 5\%\\
					Settling: 1.2s
					\vspace{1cm}
			\end{minipage}}
		}
		\label{fig:pid_response}
	}
	
	\subfigure[Fuzzy Logic Response]{
		\IfFileExists{Figures/Chapter03/fuzzy_response.jpg}{%
			\includegraphics[width=0.45\textwidth]{Figures/Chapter03/fuzzy_response.jpg}
		}{%
			\fbox{\begin{minipage}{0.4\textwidth}
					\centering
					\vspace{1cm}
					\textbf{Fuzzy Response}\\
					Overshoot: 8\%\\
					Settling: 1.8s
					\vspace{1cm}
			\end{minipage}}
		}
		\label{fig:fuzzy_response}
	}
	\hfill
	\subfigure[Neural Network Response]{
		\IfFileExists{Figures/Chapter03/neural_response.jpg}{%
			\includegraphics[width=0.45\textwidth]{Figures/Chapter03/neural_response.jpg}
		}{%
			\fbox{\begin{minipage}{0.4\textwidth}
					\centering
					\vspace{1cm}
					\textbf{Neural Response}\\
					Overshoot: 3\%\\
					Settling: 1.0s
					\vspace{1cm}
			\end{minipage}}
		}
		\label{fig:neural_response}
	}
	\caption{Comparison of control system responses: (a) PI controller, (b) PID controller, (c) Fuzzy logic controller, (d) Neural network controller}
	\label{fig:control_systems}
\end{figure}

\section{Figure in Table}
\label{sec:figure_in_table}

Table \ref{tab:converter_comparison} presents power converter topologies with integrated circuit diagrams \cite{blaabjerg2006overview}.

\begin{table}[htbp]
	\centering
	\caption{Power Converter Topology Comparison}
	\label{tab:converter_comparison}
	\begin{tabular}{|l|c|c|c|}
		\hline
		\textbf{Topology} & \textbf{Circuit Diagram} & \textbf{Efficiency} & \textbf{Cost} \\
		\hline
		Buck Converter & 
		\IfFileExists{Figures/Chapter03/buck_circuit.jpg}{%
			\includegraphics[width=3cm]{Figures/Chapter03/buck_circuit.jpg}
		}{%
			\fbox{\begin{minipage}{3cm}
					\centering
					\vspace{0.5cm}
					Buck Circuit\\
					L-C Filter
					\vspace{0.5cm}
			\end{minipage}}
		}
		& 92\% & Low \\
		\hline
		Boost Converter & 
		\IfFileExists{Figures/Chapter03/boost_circuit.jpg}{%
			\includegraphics[width=3cm]{Figures/Chapter03/boost_circuit.jpg}
		}{%
			\fbox{\begin{minipage}{3cm}
					\centering
					\vspace{0.5cm}
					Boost Circuit\\
					Step-up
					\vspace{0.5cm}
			\end{minipage}}
		}
		& 89\% & Low \\
		\hline
		Full Bridge & 
		\IfFileExists{Figures/Chapter03/bridge_circuit.jpg}{%
			\includegraphics[width=3cm]{Figures/Chapter03/bridge_circuit.jpg}
		}{%
			\fbox{\begin{minipage}{3cm}
					\centering
					\vspace{0.5cm}
					Bridge Circuit\\
					4 Switches
					\vspace{0.5cm}
			\end{minipage}}
		}
		& 95\% & High \\
		\hline
	\end{tabular}
\end{table}

\section{Simple Figure Example}
\label{sec:simple_figure}

Figure \ref{fig:power_system_diagram} shows a basic power system single-line diagram \cite{anderson1999power}.

\begin{figure}[htbp]
	\centering
	\IfFileExists{Figures/Chapter03/power_system.jpg}{%
		\includegraphics[width=0.7\textwidth]{Figures/Chapter03/power_system.jpg}
	}{%
		\fbox{\begin{minipage}{0.7\textwidth}
				\centering
				\vspace{2cm}
				\textbf{Power System Single Line Diagram}\\
				\vspace{0.5cm}
				Generator -- Transformer -- Transmission Line -- Load\\
				\vspace{0.5cm}
				[Image: power\_system.jpg not found]
				\vspace{2cm}
		\end{minipage}}
	}
	\caption{Single-line diagram of a typical power transmission system with generator, transformer, and load components}
	\label{fig:power_system_diagram}
\end{figure}

\section{IEEE Figure Standards and Citations}
\label{sec:ieee_figure_standards}

All figures follow IEEE formatting standards:

\begin{itemize}
	\item Captions placed below figures
	\item Sequential numbering (Figure 3.1, 3.2, 3.3, 3.4)
	\item Proper citations: ``Figure \ref{fig:smart_grid_architecture} shows...''
	\item Subfigures labeled (a), (b), (c), (d)
	\item Automatic file detection (.jpg, .png formats)
	\item Graceful handling of missing image files
\end{itemize}

The four figure types cover essential requirements: resizable figures for different page layouts, multiple subfigures for comparisons, integrated diagrams in tables, and basic single figures for general documentation.
%%%%%%%%%%%%%%%%%%%%%%%%%%%%%%%%%%%%%%%%%%%%%%%%%%%%%%%%%%%%%%%%%%%%%%%
% Title: Chapter 4 - Mathematical Equations with siunitx Package
% Purpose: IEEE standard mathematical notation using modern siunitx
% Compiler: pdflatex
% OS: Manjaro 
% Version: V2 (siunitx)
% Written on: July 06, 2025
% Revision Date: July 06, 2025
% Author: Kingsuk Majumdar
% Copyright (c) 2025 Kingsuk Majumdar
%%%%%%%%%%%%%%%%%%%%%%%%%%%%%%%%%%%%%%%%%%%%%%%%%%%%%%%%%%%%%%%%%%%%%%%

\chapter{Mathematical Equations and IEEE Standards}
\label{chap:math}

This chapter demonstrates IEEE standards for mathematical equations using the modern siunitx package for proper unit notation and formatting.

\section{Simple Mathematical Equation}
\label{sec:simple_equation}

Equation \ref{eq:ohms_law} presents Ohm's law, which is fundamental to electrical circuit analysis \cite{alexander2016fundamentals}. This demonstrates the standard IEEE format for mathematical equations with proper numbering and citation.

\begin{equation}
	V = I \cdot R
	\label{eq:ohms_law}
\end{equation}

where $V$ is the voltage in \unit{\volt}, $I$ is the current in \unit{\ampere}, and $R$ is the resistance in \unit{\ohm}.

\section{Multi-line Mathematical Equation}
\label{sec:multiline_equation}

The power flow equations for an electrical power system require multi-line mathematical expressions. Equations \ref{eq:power_flow_P}, \ref{eq:power_flow_Q}, and \ref{eq:power_flow_S} show the complete power flow formulation using IEEE alignment standards \cite{kundur1994power}.

\begin{align}
	P_i &= V_i \sum_{j=1}^{n} V_j \left[ G_{ij} \cos(\delta_i - \delta_j) + B_{ij} \sin(\delta_i - \delta_j) \right] \label{eq:power_flow_P} \\
	Q_i &= V_i \sum_{j=1}^{n} V_j \left[ G_{ij} \sin(\delta_i - \delta_j) - B_{ij} \cos(\delta_i - \delta_j) \right] \label{eq:power_flow_Q} \\
	S_i &= P_i + jQ_i = V_i I_i^* \label{eq:power_flow_S}
\end{align}

where:
\begin{itemize}
	\item $P_i$ is the real power injection at bus $i$ (\unit{\watt})
	\item $Q_i$ is the reactive power injection at bus $i$ (\unit{\volt\ampere})  
	\item $S_i$ is the complex power at bus $i$ (\unit{\volt\ampere})
	\item $V_i$ is the voltage magnitude at bus $i$ (\unit{\volt})
	\item $\delta_i$ is the voltage angle at bus $i$ (\unit{\radian})
	\item $G_{ij}$ is the conductance of line $i$-$j$ (\unit{\siemens})
	\item $B_{ij}$ is the susceptance of line $i$-$j$ (\unit{\siemens})
	\item $n$ is the total number of buses
\end{itemize}

\section{Long Multi-line Mathematical Equations}
\label{sec:long_equations}

For complex electrical engineering formulations, long equations often require breaking the right-hand side into multiple lines. Equation \ref{eq:long_transfer_function} demonstrates a high-order transfer function for a power electronic converter with proper IEEE line breaking \cite{mohan2003power}.

\begin{equation}
	\begin{split}
		H(s) &= \frac{K_p \cdot \omega_n^2 \cdot (1 + sT_z)}{s^4 + 2\zeta_1\omega_{n1}s^3 + \omega_{n1}^2s^2 + 2\zeta_2\omega_{n2}s + \omega_{n2}^2} \\
		&\quad \times \frac{(1 + sT_{z1})(1 + sT_{z2})}{(1 + sT_{p1})(1 + sT_{p2})(1 + sT_{p3})} \\
		&\quad \times \frac{\exp(-sT_d)}{1 + sT_f} \cdot \frac{1}{1 + \frac{s}{\omega_c}}
	\end{split}
	\label{eq:long_transfer_function}
\end{equation}

where $K_p$ is the proportional gain, $\omega_n$ is the natural frequency (\unit{\radian\per\second}), $T_z$, $T_{z1}$, $T_{z2}$ are zero time constants (\unit{\second}), $T_{p1}$, $T_{p2}$, $T_{p3}$ are pole time constants (\unit{\second}), $T_d$ is the delay time (\unit{\second}), $T_f$ is the filter time constant (\unit{\second}), and $\omega_c$ is the cutoff frequency (\unit{\radian\per\second}).

\section{Conditional Mathematical Equations}
\label{sec:conditional_equations}

Conditional equations are frequently used in electrical engineering for piecewise functions, control algorithms, and protection systems. Equation \ref{eq:pwm_switching} shows the switching function for a pulse-width modulated inverter \cite{rashid2017power}.

\begin{equation}
	S_a(t) = \begin{cases}
		1 & \text{if } v_{\text{control}}(t) > v_{\text{triangular}}(t) \\
		0 & \text{if } v_{\text{control}}(t) \leq v_{\text{triangular}}(t)
	\end{cases}
	\label{eq:pwm_switching}
\end{equation}

Another example is the fault current calculation in power systems, shown in Equation \ref{eq:fault_current}:

\begin{equation}
	I_{\text{fault}} = \begin{cases}
		\frac{V_{\text{pre-fault}}}{Z_1 + Z_2 + Z_0} & \text{if single line-to-ground fault} \\
		\frac{V_{\text{pre-fault}}}{Z_1 + Z_2} & \text{if line-to-line fault} \\
		\frac{V_{\text{pre-fault}}}{Z_1} & \text{if three-phase fault} \\
		\frac{V_{\text{pre-fault}}}{\sqrt{3}(Z_1 + Z_2 + Z_0)} & \text{if double line-to-ground fault}
	\end{cases}
	\label{eq:fault_current}
\end{equation}

where $Z_1$, $Z_2$, and $Z_0$ are the positive, negative, and zero sequence impedances respectively (\unit{\ohm}), and $V_{\text{pre-fault}}$ is the pre-fault voltage (\unit{\volt}).

For control systems, the discrete-time control law can be expressed as shown in Equation \ref{eq:discrete_control}:

\begin{equation}
	u[k] = \begin{cases}
		K_p e[k] + K_i \sum_{j=0}^{k} e[j] + K_d (e[k] - e[k-1]) & \text{if } |e[k]| > \varepsilon \\
		0 & \text{if } |e[k]| \leq \varepsilon \text{ and } k > k_{\text{settle}} \\
		u_{\text{nominal}} & \text{if system in steady-state mode}
	\end{cases}
	\label{eq:discrete_control}
\end{equation}

where $u[k]$ is the control output at sample $k$, $e[k]$ is the error signal, $K_p$, $K_i$, $K_d$ are the PID gains, $\varepsilon$ is the error threshold, and $k_{\text{settle}}$ is the settling time index.

\section{IEEE Unit Standards with siunitx}
\label{sec:ieee_units}

According to IEEE standards, units must be written in roman (upright) font, not italic, and follow specific formatting rules \cite{ieee2018style}. The siunitx package provides excellent unit formatting commands. Table \ref{tab:ieee_units} shows the correct notation for common electrical engineering units.

\textbf{siunitx Package Commands:}
\begin{itemize}
	\item \texttt{\textbackslash SI\{number\}\{unit\}} - for values with units: \SI{230}{\volt}
	\item \texttt{\textbackslash unit\{unit\}} - for units only: \unit{\hertz}
	\item \texttt{\textbackslash micro} - for micro prefix: \SI{100}{\micro\ampere}
	\item \texttt{\textbackslash ohm} - for ohm symbol: \unit{\ohm}
	\item \texttt{\textbackslash percent} - for percentage: \SI{5}{\percent}
\end{itemize}

\begin{table}[htbp]
	\centering
	\caption{IEEE Standard Unit Notation with siunitx Package}
	\label{tab:ieee_units}
	\adjustbox{width=\textwidth,center}{%
		\begin{tabular}{|l|c|c|l|}
			\hline
			\textbf{Quantity} & \textbf{Symbol} & \textbf{Unit} & \textbf{siunitx Code} \\
			\hline
			Voltage & $V$ & \unit{\volt} & \texttt{\textbackslash unit\{\textbackslash volt\}} \\
			Current & $I$ & \unit{\ampere} & \texttt{\textbackslash unit\{\textbackslash ampere\}} \\
			Resistance & $R$ & \unit{\ohm} & \texttt{\textbackslash unit\{\textbackslash ohm\}} \\
			Power & $P$ & \unit{\watt} & \texttt{\textbackslash unit\{\textbackslash watt\}} \\
			Reactive Power & $Q$ & \unit{\volt\ampere} & \texttt{\textbackslash unit\{\textbackslash volt\textbackslash ampere\}} \\
			Apparent Power & $S$ & \unit{\volt\ampere} & \texttt{\textbackslash unit\{\textbackslash volt\textbackslash ampere\}} \\
			Energy & $W$ & \unit{\watt\hour} & \texttt{\textbackslash unit\{\textbackslash watt\textbackslash hour\}} \\
			Frequency & $f$ & \unit{\hertz} & \texttt{\textbackslash unit\{\textbackslash hertz\}} \\
			Capacitance & $C$ & \unit{\farad} & \texttt{\textbackslash unit\{\textbackslash farad\}} \\
			Inductance & $L$ & \unit{\henry} & \texttt{\textbackslash unit\{\textbackslash henry\}} \\
			Magnetic Flux & $\Phi$ & \unit{\weber} & \texttt{\textbackslash unit\{\textbackslash weber\}} \\
			Magnetic Field & $B$ & \unit{\tesla} & \texttt{\textbackslash unit\{\textbackslash tesla\}} \\
			Electric Field & $E$ & \unit{\volt\per\meter} & \texttt{\textbackslash unit\{\textbackslash volt\textbackslash per\textbackslash meter\}} \\
			Conductance & $G$ & \unit{\siemens} & \texttt{\textbackslash unit\{\textbackslash siemens\}} \\
			Impedance & $Z$ & \unit{\ohm} & \texttt{\textbackslash unit\{\textbackslash ohm\}} \\
			Admittance & $Y$ & \unit{\siemens} & \texttt{\textbackslash unit\{\textbackslash siemens\}} \\
			\hline
		\end{tabular}%
	}
\end{table}

\textbf{IEEE Unit Writing Rules with siunitx Package:}
\begin{itemize}
	\item Units are written in roman font: \textbf{Correct:} \SI{10}{\volt}, \textbf{Wrong:} 10~$V$
	\item Automatic spacing: \textbf{Correct:} \SI{50}{\hertz}, \textbf{Manual:} 50~\unit{\hertz}
	\item No period after unit symbols: \textbf{Correct:} \SI{100}{\watt}, \textbf{Wrong:} 100~W.
	\item Use proper prefixes: \SI{11}{\kilo\volt}, \SI{5}{\mega\watt}, \SI{100}{\micro\ampere}, \SI{5}{\milli\henry}
	\item Complex units: \SI{230}{\volt\per\meter}, \SI{50}{\ohm\per\kilo\meter}
\end{itemize}

\section{Common Mathematical Symbols}
\label{sec:math_symbols}

Table \ref{tab:math_symbols} presents common mathematical symbols used in electrical engineering with their LaTeX notation and IEEE standard representation \cite{gratzer2016more}.

\begin{table}[htbp]
	\centering
	\caption{Common Mathematical Symbols in Electrical Engineering}
	\label{tab:math_symbols}
	\adjustbox{width=\textwidth,center}{%
		\begin{tabular}{|l|c|c|l|}
			\hline
			\textbf{Description} & \textbf{Symbol} & \textbf{LaTeX Code} & \textbf{Usage Example} \\
			\hline
			\multicolumn{4}{|c|}{\textbf{Basic Operations}} \\
			\hline
			Multiplication & $\cdot$ & \texttt{\textbackslash cdot} & $V = I \cdot R$ \\
			Division & $/$ & \texttt{/} & $f = 1/T$ \\
			Plus/minus & $\pm$ & \texttt{\textbackslash pm} & $V = \SI{230 \pm 10}{\volt}$ \\
			Proportional & $\propto$ & \texttt{\textbackslash propto} & $P \propto I^2$ \\
			Approximately & $\approx$ & \texttt{\textbackslash approx} & $\pi \approx 3.14$ \\
			\hline
			\multicolumn{4}{|c|}{\textbf{Greek Letters}} \\
			\hline
			Alpha & $\alpha$ & \texttt{\textbackslash alpha} & Attenuation constant \\
			Beta & $\beta$ & \texttt{\textbackslash beta} & Phase constant \\
			Gamma & $\gamma$ & \texttt{\textbackslash gamma} & Propagation constant \\
			Delta & $\delta, \Delta$ & \texttt{\textbackslash delta, \textbackslash Delta} & Phase angle, change \\
			Epsilon & $\varepsilon$ & \texttt{\textbackslash varepsilon} & Permittivity \\
			Theta & $\theta, \Theta$ & \texttt{\textbackslash theta, \textbackslash Theta} & Phase angle \\
			Lambda & $\lambda$ & \texttt{\textbackslash lambda} & Wavelength \\
			Mu & $\mu$ & \texttt{\textbackslash mu} & Permeability, micro \\
			Pi & $\pi$ & \texttt{\textbackslash pi} & Mathematical constant \\
			Rho & $\rho$ & \texttt{\textbackslash rho} & Resistivity \\
			Sigma & $\sigma, \Sigma$ & \texttt{\textbackslash sigma, \textbackslash Sigma} & Conductivity, summation \\
			Tau & $\tau$ & \texttt{\textbackslash tau} & Time constant \\
			Phi & $\phi, \Phi$ & \texttt{\textbackslash phi, \textbackslash Phi} & Phase, magnetic flux \\
			Omega & $\omega, \Omega$ & \texttt{\textbackslash omega, \textbackslash Omega} & Angular frequency, ohm \\
			\hline
			\multicolumn{4}{|c|}{\textbf{Complex Numbers}} \\
			\hline
			Imaginary unit & $j$ & \texttt{j} & $Z = R + jX$ \\
			Real part & $\Re$ & \texttt{\textbackslash Re} & $\Re\{Z\} = R$ \\
			Imaginary part & $\Im$ & \texttt{\textbackslash Im} & $\Im\{Z\} = X$ \\
			Magnitude & $|Z|$ & \texttt{|Z|} & $|Z| = \sqrt{R^2 + X^2}$ \\
			Angle & $\angle Z$ & \texttt{\textbackslash angle Z} & $\angle Z = \arctan(X/R)$ \\
			\hline
		\end{tabular}%
	}
\end{table}

\section{IEEE Mathematical Writing Standards}
\label{sec:ieee_math_standards}

IEEE mathematical notation standards require \cite{ieee2018style}:

\begin{itemize}
	\item \textbf{Variables:} Written in italic font ($V$, $I$, $R$)
	\item \textbf{Functions:} Written in roman font ($\sin$, $\cos$, $\log$, $\exp$)
	\item \textbf{Units:} Use siunitx package commands (\texttt{\textbackslash SI\{10\}\{\textbackslash volt\}}, \texttt{\textbackslash unit\{\textbackslash hertz\}})
	\item \textbf{Constants:} Mathematical constants in roman ($\mathrm{e}$, $\pi$)
	\item \textbf{Operators:} Proper spacing around operators ($a + b$, not $a+b$)
	\item \textbf{Subscripts/Superscripts:} Roman if descriptive ($V_{\mathrm{rms}}$), italic if variable ($V_i$)
\end{itemize}

Example of correct IEEE mathematical formatting using siunitx package:
\begin{equation}
	V_{\mathrm{rms}} = \sqrt{\frac{1}{T} \int_0^T v^2(t) \, dt} \quad \text{(\unit{\volt})}
	\label{eq:rms_voltage}
\end{equation}

or with integrated number and unit:
\begin{equation}
	V_{\mathrm{rms}} = \SI{230}{\volt} \pm \SI{10}{\percent}
	\label{eq:rms_voltage_example}
\end{equation}

The systematic application of these IEEE mathematical standards with the modern siunitx package ensures consistent, professional presentation of technical equations and expressions in electrical engineering documentation.%Chapter04_Math.tex
%%%%%%%%%%%%%%%%%%%%%%%%%%%%%%%%%%%%%%%%%%%%%%%%%%%%%%%%%%%%%%%%%%%%%%%
% Title: LaTeX Thesis Template Usage Guide Chapter
% Purpose: Comprehensive guide for using the thesis template
% Compiler: pdflatex
% OS: Manjaro 
% Version: V1
% Written on: July 07, 2025
% Revision Date: July 07, 2025
% Author: Kingsuk Majumdar
% Copyright (c) 2025 Kingsuk Majumdar
%%%%%%%%%%%%%%%%%%%%%%%%%%%%%%%%%%%%%%%%%%%%%%%%%%%%%%%%%%%%%%%%%%%%%%%

\chapter{LaTeX Thesis Template Usage Guide}
\label{ch:template_guide}
\justifying

\section{Introduction}
\label{sec:intro}

The main aim of this chapter is to provide a comprehensive guide for using the LaTeX thesis template specifically designed for Dr. B. C. Roy Engineering College. This template has been developed to streamline the thesis writing process for both undergraduate and postgraduate students while maintaining institutional formatting standards and academic presentation quality.

The template architecture follows a modular approach with clear separation between user inputs and system-level formatting commands. The primary advantage of this template lies in its automated handling of multi-student configurations, conditional rendering of content based on degree type, and professional formatting that adheres to institutional guidelines.

\section{Template Architecture and Directory Structure}
\label{sec:architecture}

The template follows a well-organized hierarchical structure that facilitates easy content management and compilation. The complete directory structure is presented below:

\begin{verbatim}
	ug-thesis-template/
	|-- main.tex                    # Main document file (User Input Section)
	|-- thesis.cls                  # LaTeX class file (Formatting Engine)
	|-- references.bib              # Bibliography database
	|-- mcode.sty                   # MATLAB code highlighting package
	|-- README.md                   # Documentation file
	|-- LICENSE.lic                 # License information
	|-- Frontmatter/
	|   |-- Declaration.tex         # Student declaration page (Dont Change it) 
	|   |-- Certificate.tex   # Supervisor approval   certificate (Dont Change it)
	|   |-- Acknowledgment.tex      # Acknowledgments section
	|   |-- Abstract.tex           # Abstract and keywords
	|   +-- Acronyms.tex           # List of abbreviations and nomenclature
	|-- Chapters/
	|   |-- Chapter01_Introduction.tex    # Introduction  chapter (MUST BE)
	|   |-- Chapter02_Literature.tex     # Literature review (MUST BE)
	|   |-- Chapter02_Table.tex          # Table examples
	|   |-- Chapter03_Figure.tex         # Figure examples
	|   |-- Chapter04_Math.tex           # Mathematical expressions
	|   |-- Chapter03_Methodology.tex    # Research methodology
	|   |-- Chapter04_Implementation.tex # Implementation details
	|   |-- Chapter05_Results.tex        # Results and analysis (MUST BE)
	|   +-- Chapter06_Conclusion.tex     # Conclusions and future work (MUST BE)
	|-- Backmatter/
	|   |-- PublicationsList.tex    # Publications by authors
	|   +-- AuthorBio.tex          # Author biographies (Strictly PG/PhD only)
	|-- Figures/
	|   |-- college_logo.png       # Institutional logo (required)
	|   |-- StudentOne_photo.jpg   # Student photograph
	|   |-- StudentTwo_photo.jpg   # Student photograph
	|   |-- StudentThree_photo.jpg # Student photograph
	|   |-- Chapter01/             # Chapter-wise figure organization
	|   |-- Chapter02/
	|   |-- Chapter03/
	|   |-- Chapter04/
	|   |-- Chapter05/
	|   +-- Chapter06/
	+-- OUTPUT/                    # Generated output files (after compilation)
	|-- main.pdf              # Final thesis document
	|-- main.aux              # Auxiliary file
	|-- main.bbl              # Bibliography file
	|-- main.blg              # Bibliography log
	|-- main.log              # Compilation log
	|-- main.toc              # Table of contents
	|-- main.lof              # List of figures
	+-- main.lot              # List of tables
\end{verbatim}

\section{Configuration and User Input Section}
\label{sec:configuration}

The template utilizes a sophisticated variable definition system within the \texttt{main.tex} file. All user-specific information is contained within the clearly marked ``USER INPUT SECTION'' which must be modified according to individual thesis requirements.

\subsection{Thesis Information Configuration}
\label{subsec:thesis_info}

The fundamental thesis parameters are defined through the following commands:

\begin{verbatim}
	%% Thesis Information
	\ThesisTitle{Long Thesis Title}
	\ShortTitle{Short Thesis Title}
	\Department{Department of Electrical Engineering}
	\College{Dr. B. C. Roy Engineering College}
	\University{Maulana Abul Kalam Azad University of Technology, West Bengal}
	\DegreeType{Bachelor of Technology (B. TECH)}
	\ThesisYear{2025}
	\ThesisMonth{May}
	\Location{Durgapur}
	\AY{2024-2025}
	\Address{Durgapur – 713206, West Bengal, India}
	\Affiliation{(An Autonomous Institute, Affiliated To MAKAUT)}
\end{verbatim}

\subsection{Project-Specific Information}
\label{subsec:project_info}

For academic projects, the following parameters must be configured:

\begin{verbatim}
	%% Project Information
	\GroupNo{Group 00}
	\PaperName{Final Year Project Stage-II}
	\PaperCode{PWEE881}
\end{verbatim}

\subsection{Student Configuration System}
\label{subsec:student_config}

The template implements a dynamic student handling system that automatically adjusts content based on the number of students specified:

\begin{verbatim}
	%% Number of Students Configuration
	\NumberOfStudents{3}  % Range: 1-3 for UG, 1 for PG
	
	%% Student Information
	\StudentOne{Pradosh Chandra Mitter}
	\RollOne{18/EE/045}
	\RegOne{184410301045}
	\EmailOne{pradosh.mitter@bcrec.ac.in}
	\PhotoOne{Figures/StudentOne_photo.jpg}
	
	\StudentTwo{Tapesh Ranjan Mitter}
	\RollTwo{18/EE/052}
	\RegTwo{184410301052}
	\EmailTwo{tapesh.mitter@bcrec.ac.in}
	\PhotoTwo{Figures/StudentTwo_photo.jpg}
	
	\StudentThree{Lalmohan Gonguly}
	\RollThree{18/EE/063}
	\RegThree{184410301063}
	\EmailThree{lalmohan.gonguly@bcrec.ac.in}
	\PhotoThree{Figures/StudentThree_photo.jpg}
\end{verbatim}

\subsection{Supervision Structure}
\label{subsec:supervision}

The template accommodates both single supervisor and co-supervisor configurations:

\begin{verbatim}
	%% Supervisor Configuration
	\HasCoSupervisor{2} % 1=supervisor only, 2=both supervisor and co-supervisor
	\Supervisor{Professor C.V. Raman}
	\SupervisorDesignation{Professor}
	\SupervisorEmail{cv.raman@bcrec.ac.in}
	\SupervisorDept{Department of Electrical Engineering}
	
	\CoSupervisor{Acharya Prafulla Chandra Ray}
	\CoSupervisorDesignation{Assistant Professor}
	\CoSupervisorEmail{pc.ray@bcrec.ac.in}
	\CoSupervisorDept{Department of Electrical Engineering}
\end{verbatim}

\section{Degree-Specific Configurations}
\label{sec:degree_config}

\subsection{Undergraduate (UG) Thesis Requirements}
\label{subsec:ug_requirements}

For undergraduate theses, the following specifications must be observed:

\begin{itemize}
	\item \textbf{Maximum Students}: 3 students per group
	\item \textbf{Author Biography}: Not included in final document
	\item \textbf{Degree Type}: Bachelor of Technology (B. TECH)
	\item \textbf{Paper Code}: PWEE881 (Final Year Project Stage-II)
\end{itemize}

The configuration for undergraduate thesis should exclude author biography by commenting out the relevant include statement:

\begin{verbatim}
	% Publications by authors
	%%%%%%%%%%%%%%%%%%%%%%%%%%%%%%%%%%%%%%%%%%%%%%%%%%%%%%%%%%%%%%%%%%%%%%%
% Title: Publications by the Authors
% Purpose: List of publications by thesis authors
% Compiler: pdflatex
% OS: Manjaro 
% Version: V1
% Written on: July 05, 2025
% Revision Date: July 05, 2025
% Author: Kingsuk Majumdar
% Copyright (c) 2025 Kingsuk Majumdar
%%%%%%%%%%%%%%%%%%%%%%%%%%%%%%%%%%%%%%%%%%%%%%%%%%%%%%%%%%%%%%%%%%%%%%%

\chapter*{Publications by the Authors}
\addcontentsline{toc}{chapter}{Publications by the Authors}
\chaptermark{Publications by the Authors}
\thispagestyle{plain}

\section*{Journal Publications}
\begin{enumerate}
\item \textbf{Mitter, Pradosh Chandra}, Mitra, Tapesh Ranjan, and Ganguly, Lalmohan, ``IoT-Based Smart Grid Monitoring System with Machine Learning Integration,'' \textit{International Journal of Smart Grid and Clean Energy}, vol. 14, no. 2, pp. 45-58, 2025. DOI: 10.12720/sgce.14.2.45-58

\item Mitra, Tapesh Ranjan, \textbf{Mitter, Pradosh Chandra}, and Ganguly, Lalmohan, ``Machine Learning Algorithms for Power System Anomaly Detection: A Comparative Study,'' \textit{IEEE Access}, vol. 13, pp. 15234-15247, 2025. DOI: 10.1109/ACCESS.2025.3456789

\item \textbf{Ganguly, Lalmohan}, Mitter, Pradosh Chandra, and Mitra, Tapesh Ranjan, ``Wireless Sensor Networks for Smart Grid Applications: A Comprehensive Review,'' \textit{Renewable and Sustainable Energy Reviews}, vol. 145, article 111098, 2025. DOI: 10.1016/j.rser.2025.111098
\end{enumerate}

\section*{Conference Publications}
\begin{enumerate}
\item \textbf{Ganguly, Lalmohan}, Mitter, Pradosh Chandra, and Mitra, Tapesh Ranjan, ``Development of Wireless Sensor Network for Real-time Grid Monitoring,'' in \textit{Proceedings of IEEE International Conference on Power Electronics and Drives}, New Delhi, India, March 2025, pp. 234-239. DOI: 10.1109/IPED.2025.9123456

\item Mitter, Pradosh Chandra, \textbf{Mitra, Tapesh Ranjan}, and Ganguly, Lalmohan, ``Performance Evaluation of ML Algorithms in Smart Grid Applications,'' in \textit{National Conference on Advances in Electrical Engineering}, Dr. B. C. Roy Engineering College, Durgapur, February 2025, pp. 67-72.

\item \textbf{Mitra, Tapesh Ranjan}, Ganguly, Lalmohan, and Mitter, Pradosh Chandra, ``IoT Security Framework for Smart Grid Systems,'' in \textit{International Conference on Cybersecurity in Smart Grid}, IIT Kharagpur, January 2025, pp. 156-161.
\end{enumerate}

\section*{Under Review}
\begin{enumerate}
\item \textbf{Mitter, Pradosh Chandra}, Mitra, Tapesh Ranjan, Ganguly, Lalmohan, and Majumdar, Kingsuk, ``Comprehensive Analysis of IoT Security in Smart Grid Systems,'' \textit{Renewable and Sustainable Energy Reviews}, Elsevier. [Under Review - Submitted December 2024]

\item Ganguly, Lalmohan, \textbf{Mitter, Pradosh Chandra}, and Mitra, Tapesh Ranjan, ``Edge Computing for Real-time Smart Grid Data Processing,'' \textit{IEEE Transactions on Smart Grid}, IEEE. [Under Review - Submitted January 2025]
\end{enumerate}

\cleardoublepage
	
	% About the authors - COMMENTED OUT FOR UG
	%\include{Backmatter/AuthorBio} % Applicable for PG/PhD ONLY
\end{verbatim}

\subsection{Postgraduate (PG) Thesis Requirements}
\label{subsec:pg_requirements}

For postgraduate theses, the specifications are:

\begin{itemize}
	\item \textbf{Number of Students}: 1 student only
	\item \textbf{Author Biography}: Mandatory inclusion
	\item \textbf{Degree Type}: Master of Technology (M. TECH) or equivalent
	\item \textbf{Enhanced Documentation}: Comprehensive literature review and methodology
\end{itemize}

The configuration for postgraduate thesis must include author biography:

\begin{verbatim}
	% Publications by authors
	%%%%%%%%%%%%%%%%%%%%%%%%%%%%%%%%%%%%%%%%%%%%%%%%%%%%%%%%%%%%%%%%%%%%%%%
% Title: Publications by the Authors
% Purpose: List of publications by thesis authors
% Compiler: pdflatex
% OS: Manjaro 
% Version: V1
% Written on: July 05, 2025
% Revision Date: July 05, 2025
% Author: Kingsuk Majumdar
% Copyright (c) 2025 Kingsuk Majumdar
%%%%%%%%%%%%%%%%%%%%%%%%%%%%%%%%%%%%%%%%%%%%%%%%%%%%%%%%%%%%%%%%%%%%%%%

\chapter*{Publications by the Authors}
\addcontentsline{toc}{chapter}{Publications by the Authors}
\chaptermark{Publications by the Authors}
\thispagestyle{plain}

\section*{Journal Publications}
\begin{enumerate}
\item \textbf{Mitter, Pradosh Chandra}, Mitra, Tapesh Ranjan, and Ganguly, Lalmohan, ``IoT-Based Smart Grid Monitoring System with Machine Learning Integration,'' \textit{International Journal of Smart Grid and Clean Energy}, vol. 14, no. 2, pp. 45-58, 2025. DOI: 10.12720/sgce.14.2.45-58

\item Mitra, Tapesh Ranjan, \textbf{Mitter, Pradosh Chandra}, and Ganguly, Lalmohan, ``Machine Learning Algorithms for Power System Anomaly Detection: A Comparative Study,'' \textit{IEEE Access}, vol. 13, pp. 15234-15247, 2025. DOI: 10.1109/ACCESS.2025.3456789

\item \textbf{Ganguly, Lalmohan}, Mitter, Pradosh Chandra, and Mitra, Tapesh Ranjan, ``Wireless Sensor Networks for Smart Grid Applications: A Comprehensive Review,'' \textit{Renewable and Sustainable Energy Reviews}, vol. 145, article 111098, 2025. DOI: 10.1016/j.rser.2025.111098
\end{enumerate}

\section*{Conference Publications}
\begin{enumerate}
\item \textbf{Ganguly, Lalmohan}, Mitter, Pradosh Chandra, and Mitra, Tapesh Ranjan, ``Development of Wireless Sensor Network for Real-time Grid Monitoring,'' in \textit{Proceedings of IEEE International Conference on Power Electronics and Drives}, New Delhi, India, March 2025, pp. 234-239. DOI: 10.1109/IPED.2025.9123456

\item Mitter, Pradosh Chandra, \textbf{Mitra, Tapesh Ranjan}, and Ganguly, Lalmohan, ``Performance Evaluation of ML Algorithms in Smart Grid Applications,'' in \textit{National Conference on Advances in Electrical Engineering}, Dr. B. C. Roy Engineering College, Durgapur, February 2025, pp. 67-72.

\item \textbf{Mitra, Tapesh Ranjan}, Ganguly, Lalmohan, and Mitter, Pradosh Chandra, ``IoT Security Framework for Smart Grid Systems,'' in \textit{International Conference on Cybersecurity in Smart Grid}, IIT Kharagpur, January 2025, pp. 156-161.
\end{enumerate}

\section*{Under Review}
\begin{enumerate}
\item \textbf{Mitter, Pradosh Chandra}, Mitra, Tapesh Ranjan, Ganguly, Lalmohan, and Majumdar, Kingsuk, ``Comprehensive Analysis of IoT Security in Smart Grid Systems,'' \textit{Renewable and Sustainable Energy Reviews}, Elsevier. [Under Review - Submitted December 2024]

\item Ganguly, Lalmohan, \textbf{Mitter, Pradosh Chandra}, and Mitra, Tapesh Ranjan, ``Edge Computing for Real-time Smart Grid Data Processing,'' \textit{IEEE Transactions on Smart Grid}, IEEE. [Under Review - Submitted January 2025]
\end{enumerate}

\cleardoublepage
	
	% About the authors - REQUIRED FOR PG
	\include{Backmatter/AuthorBio} % Applicable for PG/PhD ONLY
\end{verbatim}

\section{Compilation Methods}
\label{sec:compilation}

\subsection{Offline Compilation in Manjaro Linux}
\label{subsec:offline_compilation}

For offline compilation in Manjaro Linux environment, the following procedure should be followed:

\subsubsection{Prerequisites Installation}
\label{subsubsec:prerequisites}

\begin{verbatim}
	# Update system repositories
	sudo pacman -Syu
	
	# Install complete LaTeX distribution
	sudo pacman -S texlive-most texlive-bibtexextra
	
	# Alternative: Install full TeX Live distribution
	sudo pacman -S texlive-core texlive-bin texlive-latexextra texlive-fontsextra
\end{verbatim}

\subsubsection{Compilation Process}
\label{subsubsec:compilation_process}

Navigate to the thesis template directory and execute the following commands:

\begin{verbatim}
	# Navigate to project directory
	cd /path/to/ug-thesis-template/
	
	# Create output directory
	mkdir -p OUTPUT
	
	# Primary compilation sequence
	pdflatex main.tex
	bibtex main
	pdflatex main.tex
	pdflatex main.tex
	
	# Move generated files to OUTPUT directory
	mv main.pdf OUTPUT/
	mv *.aux *.bbl *.blg *.log *.toc *.lof *.lot OUTPUT/ 2>/dev/null || true
\end{verbatim}

\subsubsection{Advanced Compilation Options}
\label{subsubsec:advanced_compilation}

For debugging and optimization:

\begin{verbatim}
	# Compilation with detailed logging
	pdflatex -interaction=nonstopmode -file-line-error main.tex > 
	compilation.log 2>&1
	
	# Draft mode compilation (faster for testing)
	pdflatex "\def\isdraft{1}%%%%%%%%%%%%%%%%%%%%%%%%%%%%%%%%%%%%%%%%%%%%%%%%%%%%%%%%%%%%%%%%%%%%%%%
% Title: UG/PG Thesis Main File - Clean Structure
% Purpose: Main file with user inputs only
% Compiler: pdflatex
% OS: Manjaro 
% Version: V3 (Reorganized)
% Written on: July 01, 2025
% Revision Date: July 06, 2025
% Author: Kingsuk Majumdar
% Copyright (c) 2025 Kingsuk Majumdar
%%%%%%%%%%%%%%%%%%%%%%%%%%%%%%%%%%%%%%%%%%%%%%%%%%%%%%%%%%%%%%%%%%%%%%%

\documentclass{thesis}

%%%%%%%%%% USER INPUT SECTION - MODIFY THIS SECTION ONLY %%%%%%%%%%

%% Thesis Information
\ThesisTitle{Long Thesis Title}
\ShortTitle{Short Thesis Title}
\Department{Department of Electrical Engineering}
\College{Dr. B. C. Roy Engineering College}
\University{Maulana Abul Kalam Azad University of Technology, West Bengal}
\DegreeType{Bachelor of Technology (B. TECH)}
\ThesisYear{2025}
\ThesisMonth{May}
\Location{Durgapur}
\AY{2024-2025}
\Address{Durgapur – 713206, West Bengal, India}
\Affiliation{(An Autonomous Institute, Affiliated To MAKAUT)}

%% Project Information
\GroupNo{Group 00}
\PaperName{Final Year Project Stage-II}
\PaperCode{PWEE881}

%% Number of Students (1-3)
\NumberOfStudents{3}

%% Student Information
\StudentOne{Pradosh Chandra Mitter}
\RollOne{18/EE/045}
\RegOne{184410301045}
\EmailOne{pradosh.mitter@bcrec.ac.in}
\PhotoOne{Figures/StudentOne_photo.jpg}

\StudentTwo{Tapesh Ranjan Mitter}
\RollTwo{18/EE/052}
\RegTwo{184410301052}
\EmailTwo{tapesh.mitter@bcrec.ac.in}
\PhotoTwo{Figures/StudentTwo_photo.jpg}

\StudentThree{Lalmohan Gonguly}
\RollThree{18/EE/063}
\RegThree{184410301063}
\EmailThree{lalmohan.gonguly@bcrec.ac.in}
\PhotoThree{Figures/StudentThree_photo.jpg}



%% Supervisor Information
\HasCoSupervisor{2} % 1=supervisor only, 2=both supervisor and co-supervisor
\Supervisor{Professor C.V. Raman}
\SupervisorDesignation{Professor}
\SupervisorEmail{cv.raman@bcrec.ac.in}
\SupervisorDept{Department of Electrical Engineering}

\CoSupervisor{Acharya Prafulla Chandra Ray}
\CoSupervisorDesignation{Assistant Professor}
\CoSupervisorEmail{pc.ray@bcrec.ac.in}
\CoSupervisorDept{Department of Electrical Engineering}

%% Head of Department
\HoD{Prof. Srinivasa Ramanujan}
\HoDDesignation{Professor \& Head}
\HoDDept{Department of Electrical Engineering}

%%%%%%%%%% END OF USER INPUT SECTION %%%%%%%%%%

\begin{document}

% Set line spacing
\onehalfspacing

%%%%%%%%%% FRONT MATTER %%%%%%%%%%
\frontmatter

% Title page
\makefrontcover

% Copyright page
\makecopyrightpage

% Declaration and Certificate
\include{Frontmatter/Declaration}
\include{Frontmatter/Certificate}

% Acknowledgment and Abstract
\include{Frontmatter/Acknowledgment}
\include{Frontmatter/Abstract}

% Table of Contents
\setcounter{tocdepth}{4}
\tableofcontents

% List of Figures
\listoffigures

% List of Tables
\listoftables

% List of Abbreviations
\include{Frontmatter/Acronyms}

%%%%%%%%%% MAIN CONTENT %%%%%%%%%%
\mainmatter

% Setup headers for main content
\setupheaders

% Include individual chapters
\include{Chapters/Chapter01_Introduction}
\include{Chapters/Chapter02_Literature}
\include{Chapters/Chapter02_Table}
\include{Chapters/Chapter03_Figure}
\include{Chapters/Chapter04_Math}%Chapter04_Math.tex
\include{Chapters/Chapter05_TemplateGuide}
%\include{Chapters/Chapter04_Implementation}
%\include{Chapters/Chapter05_Results}
%\include{Chapters/Chapter06_Conclusion}

%%%%%%%%%% BACK MATTER %%%%%%%%%%
\backmatter

% Bibliography
\bibliographystyle{IEEEtran}
%\bibliographystyle{unsrt}
\bibliography{references}
\nocite{*}
% Publications by authors
\include{Backmatter/PublicationsList}

% About the authors
%\include{Backmatter/AuthorBio} % Applicable for UG/PhD ONLY

%%%%%%%%%% APPENDICES (Uncomment if needed) %%%%%%%%%%
%\appendix
%\include{Backmatter/AppendixA_Circuits}
%\include{Backmatter/AppendixB_Code}
%\include{Backmatter/AppendixC_Results}

\end{document}"
	
	# Shell escape mode (for external programs)
	pdflatex -shell-escape main.tex
\end{verbatim}

\subsection{Online Compilation using Overleaf Platform}
\label{subsec:online_compilation}

Overleaf provides a convenient cloud-based LaTeX editing and compilation environment. The template can be deployed on Overleaf through the following process:

\subsubsection{Project Setup on Overleaf}
\label{subsubsec:overleaf_setup}

\begin{enumerate}
	\item \textbf{Create New Project}: Access Overleaf platform and create a new blank project
	\item \textbf{Upload Template Files}: Upload all template files maintaining the directory structure
	\item \textbf{Set Compiler}: Configure project settings to use \texttt{pdfLaTeX} compiler
	\item \textbf{Bibliography Engine}: Set bibliography processor to \texttt{bibtex}
\end{enumerate}

\subsubsection{Overleaf Configuration Parameters}
\label{subsubsec:overleaf_config}

\begin{verbatim}
	% Overleaf-specific settings (add to main.tex if needed)
	\RequirePackage[utf8]{inputenc}  % Ensure UTF-8 encoding
	\RequirePackage[T1]{fontenc}     % Font encoding compatibility
\end{verbatim}

\subsubsection{Collaborative Features}
\label{subsubsec:collaborative}

Overleaf enables multi-user collaboration which is particularly beneficial for multi-student projects:

\begin{itemize}
	\item \textbf{Real-time Editing}: Multiple users can edit simultaneously
	\item \textbf{Version Control}: Automatic versioning and change tracking
	\item \textbf{Comment System}: Collaborative review and feedback mechanism
	\item \textbf{Bibliography Management}: Integrated reference management
\end{itemize}

\section{Content Development Guidelines}
\label{sec:content_development}

\subsection{Chapter Organization Strategy}
\label{subsec:chapter_organization}

Each chapter should follow a structured approach with clear objectives and logical flow:

\begin{verbatim}
	\chapter{Chapter Title}
	\label{ch:chaptersymbol}
	\justifying
	
	% Chapter introduction
	% Literature review (if applicable)
	% Methodology description
	% Results presentation
	% Chapter summary
\end{verbatim}

\subsection{Figure and Table Management}
\label{subsec:figure_table}

The template provides automated figure and table handling with proper referencing:

\begin{verbatim}
	\begin{figure}[H]
		\centering
		\includegraphics[width=0.8\textwidth]{Chapter01/figure_name.png}
		\caption{Descriptive caption for the figure}
		\label{fig:figurelabel}
	\end{figure}
\end{verbatim}

\subsection{Mathematical Expression Formatting}
\label{subsec:math_formatting}

For electrical engineering applications, mathematical expressions are formatted using enhanced packages:

\begin{verbatim}
	\begin{equation}
		P = V \cdot I \cdot \cos(\phi)
		\label{eq:power}
	\end{equation}
\end{verbatim}

\section{Quality Assurance and Best Practices}
\label{sec:quality_assurance}

\subsection{File Organization Recommendations}
\label{subsec:file_organization}

To maintain template integrity and facilitate collaboration, the following practices should be observed:

\begin{enumerate}
	\item \textbf{Consistent Naming}: Use descriptive file names with chapter prefixes
	\item \textbf{Image Resolution}: Maintain high-resolution images (300 DPI minimum)
	\item \textbf{Backup Strategy}: Regular backup of work using version control systems
	\item \textbf{Validation Testing}: Periodic compilation testing to identify issues early
\end{enumerate}

\subsection{Common Error Resolution}
\label{subsec:error_resolution}

Typical compilation errors and their solutions:

\begin{itemize}
	\item \textbf{Missing Packages}: Install required packages using package manager
	\item \textbf{File Path Issues}: Verify relative paths for figures and includes
	\item \textbf{Encoding Problems}: Ensure UTF-8 encoding for all text files
	\item \textbf{Bibliography Errors}: Check reference format and .bib file syntax
\end{itemize}

\section{Performance Optimization}
\label{sec:performance}

For large documents with numerous figures and references, compilation performance can be optimized through:

\begin{verbatim}
	% Draft mode for faster compilation during writing
	\documentclass[draft]{thesis}
	
	% Selective chapter compilation
	%\includeonly{Chapters/Chapter01_Introduction}
\end{verbatim}

\section{Third-Party Components and Acknowledgments}
\label{sec:acknowledgments}

This template incorporates several third-party components and packages that enhance its functionality and appearance. Proper attribution and licensing information for these components is provided below:

\subsection{MATLAB Code Highlighting}
\label{subsec:mcode}

The template includes the \texttt{mcode.sty} package developed by Florian Knorn for MATLAB code syntax highlighting. This package provides professional formatting for MATLAB code snippets within LaTeX documents. The \texttt{mcode.sty} package is distributed under the BSD License and can be used for academic and commercial purposes.

To use MATLAB code highlighting in your thesis:

\begin{verbatim}
	\begin{lstlisting}[style=Matlab-editor]
		function result = myFunction(input)
		% Your MATLAB code here
		result = input * 2;
		fprintf('Result: %f\n', result);
		end
	\end{lstlisting}
\end{verbatim}

\subsection{Template Availability and Distribution}
\label{subsec:template_availability}

This LaTeX thesis template is made available through multiple platforms to ensure easy access and collaboration:

\subsubsection{GitHub Repository}
The complete template source code, documentation, and version history are maintained in the GitHub repository:

\textbf{GitHub Link}: \texttt{[Repository link will be added here]}

The GitHub repository provides:
\begin{itemize}
	\item Complete source code with version control
	\item Issue tracking and bug reports
	\item Collaborative development environment
	\item Release management and downloads
\end{itemize}

\subsubsection{Overleaf Template}
For users preferring online LaTeX editing, the template is also available as an Overleaf template:

\textbf{Overleaf Link}: \texttt{[Overleaf template link will be added here]}

The Overleaf version offers:
\begin{itemize}
	\item One-click template import
	\item Collaborative editing capabilities
	\item Automatic compilation and preview
	\item No local LaTeX installation required
\end{itemize}

\section{Conclusion}
\label{sec:conclusion}

This chapter has provided a comprehensive overview of the LaTeX thesis template usage for Dr. B. C. Roy Engineering College. The template's modular architecture and automated formatting capabilities significantly reduce the formatting overhead, allowing students to focus on content development rather than document structure.

The distinction between undergraduate and postgraduate requirements has been clearly delineated, with specific guidelines for student numbers and author biography inclusion. Both offline compilation in Manjaro Linux and online compilation through Overleaf have been detailed to accommodate different working preferences and technical environments.

Through proper utilization of this template, students can produce professional-quality thesis documents that adhere to institutional standards while maintaining consistency across different projects and departments.

\section{License Information}
\label{sec:license}

\subsection{MIT License}
\label{subsec:mit_license}

This LaTeX thesis template is released under the MIT License, which allows for maximum flexibility in usage, modification, and distribution. The complete license text is provided below:

\begin{verbatim}
	MIT License
	
	Copyright (c) 2025 Kingsuk Majumdar
	
	Permission is hereby granted, free of charge, to any person obtaining a copy
	of this software and associated documentation files (the "Software"), to deal
	in the Software without restriction, including without limitation the rights
	to use, copy, modify, merge, publish, distribute, sublicense, and/or sell
	copies of the Software, and to permit persons to whom the Software is
	furnished to do so, subject to the following conditions:
	
	The above copyright notice and this permission notice shall be included in all
	copies or substantial portions of the Software.
	
	THE SOFTWARE IS PROVIDED "AS IS", WITHOUT WARRANTY OF ANY KIND, EXPRESS OR
	IMPLIED, INCLUDING BUT NOT LIMITED TO THE WARRANTIES OF MERCHANTABILITY,
	FITNESS FOR A PARTICULAR PURPOSE AND NONINFRINGEMENT. IN NO EVENT SHALL THE
	AUTHORS OR COPYRIGHT HOLDERS BE LIABLE FOR ANY CLAIM, DAMAGES OR OTHER
	LIABILITY, WHETHER IN AN ACTION OF CONTRACT, TORT OR OTHERWISE, ARISING FROM,
	OUT OF OR IN CONNECTION WITH THE SOFTWARE OR THE USE OR OTHER DEALINGS IN THE
	SOFTWARE.
\end{verbatim}

\subsection{Usage Terms}
\label{subsec:usage_terms}

Under the MIT License, users are granted the following rights:

\begin{itemize}
	\item \textbf{Commercial Use}: The template may be used for commercial purposes
	\item \textbf{Modification}: Users may modify the template to suit their requirements
	\item \textbf{Distribution}: The template may be distributed freely
	\item \textbf{Private Use}: Private usage is permitted without restriction
\end{itemize}

The only requirement is the inclusion of the copyright notice and license text in any distributions of the template or substantial portions thereof.

\subsection{Third-Party License Compliance}
\label{subsec:third_party_licenses}

This template incorporates third-party components with their respective licenses:

\begin{itemize}
	\item \textbf{mcode.sty}: BSD License (Florian Knorn)
	\item \textbf{Standard LaTeX Packages}: Various open-source licenses
	\item \textbf{TeX Live Distribution}: TeX Users Group License
\end{itemize}

All third-party components are used in compliance with their respective licensing terms, and users should ensure continued compliance when modifying or redistributing the template.
%\include{Chapters/Chapter04_Implementation}
%\include{Chapters/Chapter05_Results}
%\include{Chapters/Chapter06_Conclusion}

%%%%%%%%%% BACK MATTER %%%%%%%%%%
\backmatter

% Bibliography
\bibliographystyle{IEEEtran}
%\bibliographystyle{unsrt}
\bibliography{references}
\nocite{*}
% Publications by authors
%%%%%%%%%%%%%%%%%%%%%%%%%%%%%%%%%%%%%%%%%%%%%%%%%%%%%%%%%%%%%%%%%%%%%%%
% Title: Publications by the Authors
% Purpose: List of publications by thesis authors
% Compiler: pdflatex
% OS: Manjaro 
% Version: V1
% Written on: July 05, 2025
% Revision Date: July 05, 2025
% Author: Kingsuk Majumdar
% Copyright (c) 2025 Kingsuk Majumdar
%%%%%%%%%%%%%%%%%%%%%%%%%%%%%%%%%%%%%%%%%%%%%%%%%%%%%%%%%%%%%%%%%%%%%%%

\chapter*{Publications by the Authors}
\addcontentsline{toc}{chapter}{Publications by the Authors}
\chaptermark{Publications by the Authors}
\thispagestyle{plain}

\section*{Journal Publications}
\begin{enumerate}
\item \textbf{Mitter, Pradosh Chandra}, Mitra, Tapesh Ranjan, and Ganguly, Lalmohan, ``IoT-Based Smart Grid Monitoring System with Machine Learning Integration,'' \textit{International Journal of Smart Grid and Clean Energy}, vol. 14, no. 2, pp. 45-58, 2025. DOI: 10.12720/sgce.14.2.45-58

\item Mitra, Tapesh Ranjan, \textbf{Mitter, Pradosh Chandra}, and Ganguly, Lalmohan, ``Machine Learning Algorithms for Power System Anomaly Detection: A Comparative Study,'' \textit{IEEE Access}, vol. 13, pp. 15234-15247, 2025. DOI: 10.1109/ACCESS.2025.3456789

\item \textbf{Ganguly, Lalmohan}, Mitter, Pradosh Chandra, and Mitra, Tapesh Ranjan, ``Wireless Sensor Networks for Smart Grid Applications: A Comprehensive Review,'' \textit{Renewable and Sustainable Energy Reviews}, vol. 145, article 111098, 2025. DOI: 10.1016/j.rser.2025.111098
\end{enumerate}

\section*{Conference Publications}
\begin{enumerate}
\item \textbf{Ganguly, Lalmohan}, Mitter, Pradosh Chandra, and Mitra, Tapesh Ranjan, ``Development of Wireless Sensor Network for Real-time Grid Monitoring,'' in \textit{Proceedings of IEEE International Conference on Power Electronics and Drives}, New Delhi, India, March 2025, pp. 234-239. DOI: 10.1109/IPED.2025.9123456

\item Mitter, Pradosh Chandra, \textbf{Mitra, Tapesh Ranjan}, and Ganguly, Lalmohan, ``Performance Evaluation of ML Algorithms in Smart Grid Applications,'' in \textit{National Conference on Advances in Electrical Engineering}, Dr. B. C. Roy Engineering College, Durgapur, February 2025, pp. 67-72.

\item \textbf{Mitra, Tapesh Ranjan}, Ganguly, Lalmohan, and Mitter, Pradosh Chandra, ``IoT Security Framework for Smart Grid Systems,'' in \textit{International Conference on Cybersecurity in Smart Grid}, IIT Kharagpur, January 2025, pp. 156-161.
\end{enumerate}

\section*{Under Review}
\begin{enumerate}
\item \textbf{Mitter, Pradosh Chandra}, Mitra, Tapesh Ranjan, Ganguly, Lalmohan, and Majumdar, Kingsuk, ``Comprehensive Analysis of IoT Security in Smart Grid Systems,'' \textit{Renewable and Sustainable Energy Reviews}, Elsevier. [Under Review - Submitted December 2024]

\item Ganguly, Lalmohan, \textbf{Mitter, Pradosh Chandra}, and Mitra, Tapesh Ranjan, ``Edge Computing for Real-time Smart Grid Data Processing,'' \textit{IEEE Transactions on Smart Grid}, IEEE. [Under Review - Submitted January 2025]
\end{enumerate}

\cleardoublepage

% About the authors
%\include{Backmatter/AuthorBio} % Applicable for UG/PhD ONLY

%%%%%%%%%% APPENDICES (Uncomment if needed) %%%%%%%%%%
%\appendix
%\include{Backmatter/AppendixA_Circuits}
%\include{Backmatter/AppendixB_Code}
%\include{Backmatter/AppendixC_Results}

\end{document}"
	
	# Shell escape mode (for external programs)
	pdflatex -shell-escape main.tex
\end{verbatim}

\subsection{Online Compilation using Overleaf Platform}
\label{subsec:online_compilation}

Overleaf provides a convenient cloud-based LaTeX editing and compilation environment. The template can be deployed on Overleaf through the following process:

\subsubsection{Project Setup on Overleaf}
\label{subsubsec:overleaf_setup}

\begin{enumerate}
	\item \textbf{Create New Project}: Access Overleaf platform and create a new blank project
	\item \textbf{Upload Template Files}: Upload all template files maintaining the directory structure
	\item \textbf{Set Compiler}: Configure project settings to use \texttt{pdfLaTeX} compiler
	\item \textbf{Bibliography Engine}: Set bibliography processor to \texttt{bibtex}
\end{enumerate}

\subsubsection{Overleaf Configuration Parameters}
\label{subsubsec:overleaf_config}

\begin{verbatim}
	% Overleaf-specific settings (add to main.tex if needed)
	\RequirePackage[utf8]{inputenc}  % Ensure UTF-8 encoding
	\RequirePackage[T1]{fontenc}     % Font encoding compatibility
\end{verbatim}

\subsubsection{Collaborative Features}
\label{subsubsec:collaborative}

Overleaf enables multi-user collaboration which is particularly beneficial for multi-student projects:

\begin{itemize}
	\item \textbf{Real-time Editing}: Multiple users can edit simultaneously
	\item \textbf{Version Control}: Automatic versioning and change tracking
	\item \textbf{Comment System}: Collaborative review and feedback mechanism
	\item \textbf{Bibliography Management}: Integrated reference management
\end{itemize}

\section{Content Development Guidelines}
\label{sec:content_development}

\subsection{Chapter Organization Strategy}
\label{subsec:chapter_organization}

Each chapter should follow a structured approach with clear objectives and logical flow:

\begin{verbatim}
	\chapter{Chapter Title}
	\label{ch:chaptersymbol}
	\justifying
	
	% Chapter introduction
	% Literature review (if applicable)
	% Methodology description
	% Results presentation
	% Chapter summary
\end{verbatim}

\subsection{Figure and Table Management}
\label{subsec:figure_table}

The template provides automated figure and table handling with proper referencing:

\begin{verbatim}
	\begin{figure}[H]
		\centering
		\includegraphics[width=0.8\textwidth]{Chapter01/figure_name.png}
		\caption{Descriptive caption for the figure}
		\label{fig:figurelabel}
	\end{figure}
\end{verbatim}

\subsection{Mathematical Expression Formatting}
\label{subsec:math_formatting}

For electrical engineering applications, mathematical expressions are formatted using enhanced packages:

\begin{verbatim}
	\begin{equation}
		P = V \cdot I \cdot \cos(\phi)
		\label{eq:power}
	\end{equation}
\end{verbatim}

\section{Quality Assurance and Best Practices}
\label{sec:quality_assurance}

\subsection{File Organization Recommendations}
\label{subsec:file_organization}

To maintain template integrity and facilitate collaboration, the following practices should be observed:

\begin{enumerate}
	\item \textbf{Consistent Naming}: Use descriptive file names with chapter prefixes
	\item \textbf{Image Resolution}: Maintain high-resolution images (300 DPI minimum)
	\item \textbf{Backup Strategy}: Regular backup of work using version control systems
	\item \textbf{Validation Testing}: Periodic compilation testing to identify issues early
\end{enumerate}

\subsection{Common Error Resolution}
\label{subsec:error_resolution}

Typical compilation errors and their solutions:

\begin{itemize}
	\item \textbf{Missing Packages}: Install required packages using package manager
	\item \textbf{File Path Issues}: Verify relative paths for figures and includes
	\item \textbf{Encoding Problems}: Ensure UTF-8 encoding for all text files
	\item \textbf{Bibliography Errors}: Check reference format and .bib file syntax
\end{itemize}

\section{Performance Optimization}
\label{sec:performance}

For large documents with numerous figures and references, compilation performance can be optimized through:

\begin{verbatim}
	% Draft mode for faster compilation during writing
	\documentclass[draft]{thesis}
	
	% Selective chapter compilation
	%\includeonly{Chapters/Chapter01_Introduction}
\end{verbatim}

\section{Third-Party Components and Acknowledgments}
\label{sec:acknowledgments}

This template incorporates several third-party components and packages that enhance its functionality and appearance. Proper attribution and licensing information for these components is provided below:

\subsection{MATLAB Code Highlighting}
\label{subsec:mcode}

The template includes the \texttt{mcode.sty} package developed by Florian Knorn for MATLAB code syntax highlighting. This package provides professional formatting for MATLAB code snippets within LaTeX documents. The \texttt{mcode.sty} package is distributed under the BSD License and can be used for academic and commercial purposes.

To use MATLAB code highlighting in your thesis:

\begin{verbatim}
	\begin{lstlisting}[style=Matlab-editor]
		function result = myFunction(input)
		% Your MATLAB code here
		result = input * 2;
		fprintf('Result: %f\n', result);
		end
	\end{lstlisting}
\end{verbatim}

\subsection{Template Availability and Distribution}
\label{subsec:template_availability}

This LaTeX thesis template is made available through multiple platforms to ensure easy access and collaboration:

\subsubsection{GitHub Repository}
The complete template source code, documentation, and version history are maintained in the GitHub repository:

\textbf{GitHub Link}: \texttt{[Repository link will be added here]}

The GitHub repository provides:
\begin{itemize}
	\item Complete source code with version control
	\item Issue tracking and bug reports
	\item Collaborative development environment
	\item Release management and downloads
\end{itemize}

\subsubsection{Overleaf Template}
For users preferring online LaTeX editing, the template is also available as an Overleaf template:

\textbf{Overleaf Link}: \texttt{[Overleaf template link will be added here]}

The Overleaf version offers:
\begin{itemize}
	\item One-click template import
	\item Collaborative editing capabilities
	\item Automatic compilation and preview
	\item No local LaTeX installation required
\end{itemize}

\section{Conclusion}
\label{sec:conclusion}

This chapter has provided a comprehensive overview of the LaTeX thesis template usage for Dr. B. C. Roy Engineering College. The template's modular architecture and automated formatting capabilities significantly reduce the formatting overhead, allowing students to focus on content development rather than document structure.

The distinction between undergraduate and postgraduate requirements has been clearly delineated, with specific guidelines for student numbers and author biography inclusion. Both offline compilation in Manjaro Linux and online compilation through Overleaf have been detailed to accommodate different working preferences and technical environments.

Through proper utilization of this template, students can produce professional-quality thesis documents that adhere to institutional standards while maintaining consistency across different projects and departments.

\section{License Information}
\label{sec:license}

\subsection{MIT License}
\label{subsec:mit_license}

This LaTeX thesis template is released under the MIT License, which allows for maximum flexibility in usage, modification, and distribution. The complete license text is provided below:

\begin{verbatim}
	MIT License
	
	Copyright (c) 2025 Kingsuk Majumdar
	
	Permission is hereby granted, free of charge, to any person obtaining a copy
	of this software and associated documentation files (the "Software"), to deal
	in the Software without restriction, including without limitation the rights
	to use, copy, modify, merge, publish, distribute, sublicense, and/or sell
	copies of the Software, and to permit persons to whom the Software is
	furnished to do so, subject to the following conditions:
	
	The above copyright notice and this permission notice shall be included in all
	copies or substantial portions of the Software.
	
	THE SOFTWARE IS PROVIDED "AS IS", WITHOUT WARRANTY OF ANY KIND, EXPRESS OR
	IMPLIED, INCLUDING BUT NOT LIMITED TO THE WARRANTIES OF MERCHANTABILITY,
	FITNESS FOR A PARTICULAR PURPOSE AND NONINFRINGEMENT. IN NO EVENT SHALL THE
	AUTHORS OR COPYRIGHT HOLDERS BE LIABLE FOR ANY CLAIM, DAMAGES OR OTHER
	LIABILITY, WHETHER IN AN ACTION OF CONTRACT, TORT OR OTHERWISE, ARISING FROM,
	OUT OF OR IN CONNECTION WITH THE SOFTWARE OR THE USE OR OTHER DEALINGS IN THE
	SOFTWARE.
\end{verbatim}

\subsection{Usage Terms}
\label{subsec:usage_terms}

Under the MIT License, users are granted the following rights:

\begin{itemize}
	\item \textbf{Commercial Use}: The template may be used for commercial purposes
	\item \textbf{Modification}: Users may modify the template to suit their requirements
	\item \textbf{Distribution}: The template may be distributed freely
	\item \textbf{Private Use}: Private usage is permitted without restriction
\end{itemize}

The only requirement is the inclusion of the copyright notice and license text in any distributions of the template or substantial portions thereof.

\subsection{Third-Party License Compliance}
\label{subsec:third_party_licenses}

This template incorporates third-party components with their respective licenses:

\begin{itemize}
	\item \textbf{mcode.sty}: BSD License (Florian Knorn)
	\item \textbf{Standard LaTeX Packages}: Various open-source licenses
	\item \textbf{TeX Live Distribution}: TeX Users Group License
\end{itemize}

All third-party components are used in compliance with their respective licensing terms, and users should ensure continued compliance when modifying or redistributing the template.