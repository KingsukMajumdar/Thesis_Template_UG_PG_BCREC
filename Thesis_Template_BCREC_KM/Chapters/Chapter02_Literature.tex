
\chapter{Literature Review}

\section{Introduction}
This chapter presents a comprehensive review of existing literature related to the development of smart grid monitoring systems. The review is divided into thematic sections covering smart grid architectures, IoT integration, and machine learning applications. The aim is to understand the evolution, current advancements, and limitations of prior work to justify the scope of this research.

\section{Smart Grid and Monitoring Systems}
Smart grids represent the modernization of traditional electrical grids by incorporating advanced communication and automation technologies. Various studies have addressed the implementation challenges and advantages of smart grids in power systems \cite{ieee2030,fang2012smart}. Effective monitoring systems are essential for fault detection, real-time decision-making, and load management.

\section{Integration of IoT in Smart Grids}
The Internet of Things (IoT) enables connectivity between sensors, meters, and control systems, facilitating real-time data acquisition and remote monitoring \cite{zanella2014internet}. Several researchers have explored low-cost IoT-based architectures for distributed grid monitoring \cite{ghosh2017iot}, emphasizing the use of microcontrollers, wireless protocols, and cloud platforms.

\section{Machine Learning Applications}
Machine learning (ML) offers predictive and adaptive capabilities in grid analysis. Techniques such as support vector machines, decision trees, and neural networks have been used for load forecasting, fault classification, and energy consumption optimization \cite{mohamed2019machine, singh2020review}. The integration of ML with IoT enhances the intelligence and automation of smart grids.

\section{Comparative Analysis of Related Work}
Table~\ref{tab:relatedwork} summarizes key contributions in literature, comparing methods, tools used, and performance metrics.

\begin{table}[H]
\centering
\caption{Comparison of Selected Literature on Smart Grid Monitoring}
\label{tab:relatedwork}
\begin{tabular}{|p{3.5cm}|p{3cm}|p{3cm}|p{3cm}|}
\hline
\textbf{Author(s)} & \textbf{Technology Used} & \textbf{Focus Area} & \textbf{Remarks} \\
\hline
Fang et al. (2012) \cite{fang2012smart} & Communication and Security & Smart Grid Framework & Early overview of challenges and architecture \\
\hline
Zanella et al. (2014) \cite{zanella2014internet} & IoT Architecture & Urban IoT for Smart Cities & Demonstrated scalability and cost-effectiveness \\
\hline
Mohamed et al. (2019) \cite{mohamed2019machine} & ML Algorithms & Load Forecasting & High accuracy using ANN \\
\hline
Singh et al. (2020) \cite{singh2020review} & Hybrid ML Models & Fault Detection & Emphasized real-time learning models \\
\hline
\end{tabular}
\end{table}

\section{Research Gaps Identified}
From the literature, several gaps have been identified:
\begin{itemize}
    \item Lack of integrated systems combining both IoT and ML for comprehensive monitoring.
    \item Limited real-time deployment and testing on live grid systems.
    \item Data privacy and security remain less addressed in existing IoT-based models.
\end{itemize}

\section{Summary}
The literature demonstrates promising advancements in smart grid monitoring through IoT and ML. However, challenges in scalability, real-time performance, and integration offer significant scope for this research. The next chapter will elaborate on the methodology adopted in this work.

