%%%%%%%%%%%%%%%%%%%%%%%%%%%%%%%%%%%%%%%%%%%%%%%%%%%%%%%%%%%%%%%%%%%%%%%
% Title: Abstract Page
% Purpose: Abstract for UG thesis with improved formatting
% Compiler: pdflatex
% OS: Manjaro 
% Version: V2
% Written on: July 05, 2025
% Revision Date: July 05, 2025
% Author: Kingsuk Majumdar
% Copyright (c) 2025 Kingsuk Majumdar
%%%%%%%%%%%%%%%%%%%%%%%%%%%%%%%%%%%%%%%%%%%%%%%%%%%%%%%%%%%%%%%%%%%%%%%

% Put this file in Frontmatter/Abstract.tex

\chapter*{Abstract}
\addcontentsline{toc}{chapter}{Abstract}
\chaptermark{Abstract}
\thispagestyle{plain}

With the rapid advancement in technology and increasing demand for efficient power management, smart grid systems have become essential for modern electrical infrastructure. This project presents the development of a comprehensive smart grid monitoring system that integrates Internet of Things (IoT) technology with machine learning algorithms to enhance grid reliability, efficiency, and sustainability.

The proposed system utilizes various sensors and IoT devices to collect real-time data from different components of the electrical grid including voltage, current, frequency, and power quality parameters. The collected data is transmitted through wireless communication protocols to a central monitoring station where machine learning algorithms process and analyze the information to detect anomalies, predict failures, and optimize grid operations.

The machine learning component employs artificial neural networks and support vector machines to classify normal and abnormal grid conditions. The system also incorporates predictive maintenance capabilities using time-series analysis and regression techniques to forecast equipment failures and schedule maintenance activities proactively.

A user-friendly web-based interface has been developed to visualize real-time grid status, historical trends, and alert notifications. The system also includes automated control features that can respond to critical situations by adjusting load distribution or isolating faulty sections.

Simulation results demonstrate that the proposed system can effectively monitor grid parameters with 95\% accuracy in anomaly detection and reduce downtime by 30\% through predictive maintenance. The IoT-based architecture ensures scalability and cost-effectiveness, making it suitable for implementation in both urban and rural electrical networks.

The project contributes to the advancement of smart grid technology and provides a foundation for future research in intelligent power system monitoring and control. The developed system addresses the critical need for real-time monitoring and predictive maintenance in modern electrical grids, offering significant improvements in reliability and operational efficiency.

\cleardoublepage