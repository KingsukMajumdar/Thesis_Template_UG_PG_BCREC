%%%%%%%%%%%%%%%%%%%%%%%%%%%%%%%%%%%%%%%%%%%%%%%%%%%%%%%%%%%%%%%%%%%%%%%
% Title: Appendix A - Circuit Diagrams and Schematics
% Purpose: Circuit diagrams and schematics for smart grid monitoring system
% Compiler: pdflatex
% OS: Manjaro 
% Version: V1
% Written on: July 05, 2025
% Revision Date: July 05, 2025
% Author: Kingsuk Majumdar
% Copyright (c) 2025 Kingsuk Majumdar
%%%%%%%%%%%%%%%%%%%%%%%%%%%%%%%%%%%%%%%%%%%%%%%%%%%%%%%%%%%%%%%%%%%%%%%

\chapter{Circuit Diagrams and Schematics}
\label{chap:circuits}

\section{Sensor Interface Circuits}
\label{sec:sensor_circuits}

This section presents the detailed circuit diagrams for various sensor interfaces used in the smart grid monitoring system.

\subsection{Voltage Sensor Interface Circuit}
\label{subsec:voltage_sensor_circuit}

The voltage sensor interface circuit is designed to safely measure AC voltages up to 400V while providing electrical isolation and appropriate scaling for the microcontroller ADC input.

\begin{figure}[H]
\centering
\includegraphics[width=0.9\textwidth]{voltage_sensor_circuit}
\caption{Voltage Sensor Interface Circuit Diagram}
\label{fig:voltage_sensor_circuit}
\end{figure}

\textbf{Circuit Components:}
\begin{itemize}
\item \textbf{T1:} Step-down transformer (400V:12V, 2VA)
\item \textbf{R1, R2:} Voltage divider resistors (10k$\Omega$, 1\% tolerance)
\item \textbf{C1:} Filter capacitor (1$\mu$F, ceramic)
\item \textbf{D1, D2:} Protection diodes (1N4148)
\item \textbf{U1:} Operational amplifier (LM358)
\item \textbf{R3, R4:} Feedback resistors for unity gain buffer
\item \textbf{C2:} Power supply decoupling capacitor (100nF)
\end{itemize}

\textbf{Design Specifications:}
\begin{itemize}
\item Input voltage range: 0-400V AC
\item Output voltage range: 0-3.3V DC
\item Accuracy: $\pm$1\% of full scale
\item Bandwidth: DC to 1kHz
\item Isolation voltage: 2.5kV
\end{itemize}

\subsection{Current Sensor Interface Circuit}
\label{subsec:current_sensor_circuit}

The current sensor interface uses a split-core current transformer to provide non-invasive current measurement with galvanic isolation.

\begin{figure}[H]
\centering
\includegraphics[width=0.9\textwidth]{current_sensor_circuit}
\caption{Current Sensor Interface Circuit Diagram}
\label{fig:current_sensor_circuit}
\end{figure}

\textbf{Circuit Components:}
\begin{itemize}
\item \textbf{CT1:} Split-core current transformer (100A:1A ratio)
\item \textbf{R5:} Burden resistor (1$\Omega$, 2W)
\item \textbf{R6, R7:} Input protection resistors (100$\Omega$)
\item \textbf{C3, C4:} AC coupling capacitors (10$\mu$F, tantalum)
\item \textbf{U2:} Instrumentation amplifier (INA128)
\item \textbf{R8:} Gain setting resistor (499$\Omega$, 0.1\%)
\item \textbf{R9, R10:} Bias resistors (10k$\Omega$)
\end{itemize}

\textbf{Design Specifications:}
\begin{itemize}
\item Input current range: 0-100A AC
\item Output voltage range: 0-3.3V DC
\item Accuracy: $\pm$0.5\% of reading
\item Frequency response: 45-65Hz
\item Phase accuracy: $\pm$1 degree
\end{itemize}

\subsection{Temperature Sensor Interface Circuit}
\label{subsec:temperature_sensor_circuit}

The temperature sensor interface provides accurate temperature measurement for equipment monitoring and environmental conditions.

\begin{figure}[H]
\centering
\includegraphics[width=0.8\textwidth]{temperature_sensor_circuit}
\caption{Temperature Sensor Interface Circuit Diagram}
\label{fig:temperature_sensor_circuit}
\end{figure}

\textbf{Circuit Components:}
\begin{itemize}
\item \textbf{U3:} Digital temperature sensor (DS18B20)
\item \textbf{R11:} Pull-up resistor (4.7k$\Omega$)
\item \textbf{C5:} Power supply filtering capacitor (100nF)
\item \textbf{J1:} 3-pin connector for sensor cable
\end{itemize}

\textbf{Design Specifications:}
\begin{itemize}
\item Temperature range: -20°C to +85°C
\item Resolution: 0.0625°C (12-bit)
\item Accuracy: $\pm$0.5°C
\item Interface: 1-Wire digital communication
\item Power consumption: 1mA active, 1$\mu$A standby
\end{itemize}

\section{Communication Module Circuits}
\label{sec:communication_circuits}

\subsection{WiFi Communication Module}
\label{subsec:wifi_module_circuit}

The WiFi communication module provides wireless connectivity for data transmission to the central monitoring system.

\begin{figure}[H]
\centering
\includegraphics[width=0.9\textwidth]{wifi_module_circuit}
\caption{WiFi Communication Module Circuit Diagram}
\label{fig:wifi_module_circuit}
\end{figure}

\textbf{Circuit Components:}
\begin{itemize}
\item \textbf{U4:} ESP32 WiFi/Bluetooth microcontroller
\item \textbf{L1:} RF inductor for antenna matching (2.2nH)
\item \textbf{C6, C7:} Antenna matching capacitors (1pF, 2.2pF)
\item \textbf{ANT1:} 2.4GHz PCB antenna
\item \textbf{R12, R13:} GPIO pull-up resistors (10k$\Omega$)
\item \textbf{C8, C9:} Power supply decoupling capacitors (100nF, 10$\mu$F)
\item \textbf{Y1:} Crystal oscillator (40MHz)
\item \textbf{C10, C11:} Crystal load capacitors (22pF)
\end{itemize}

\textbf{Design Specifications:}
\begin{itemize}
\item WiFi standard: IEEE 802.11 b/g/n
\item Frequency range: 2.4GHz ISM band
\item Transmission power: +20dBm maximum
\item Receiver sensitivity: -98dBm
\item Data rate: Up to 150Mbps
\item Range: Up to 100m (line of sight)
\end{itemize}

\subsection{LoRa Communication Module}
\label{subsec:lora_module_circuit}

The LoRa module provides long-range, low-power communication for remote sensor deployments.

\begin{figure}[H]
\centering
\includegraphics[width=0.9\textwidth]{lora_module_circuit}
\caption{LoRa Communication Module Circuit Diagram}
\label{fig:lora_module_circuit}
\end{figure}

\textbf{Circuit Components:}
\begin{itemize}
\item \textbf{U5:} SX1276 LoRa transceiver
\item \textbf{X1:} 32MHz TCXO (temperature compensated crystal oscillator)
\item \textbf{L2, L3:} RF matching inductors (18nH, 27nH)
\item \textbf{C12-C15:} RF matching and DC blocking capacitors
\item \textbf{ANT2:} 915MHz external antenna connector
\item \textbf{R14-R17:} GPIO and control resistors
\end{itemize}

\textbf{Design Specifications:}
\begin{itemize}
\item Frequency bands: 433MHz, 868MHz, 915MHz
\item Modulation: LoRa spread spectrum
\item Transmission power: +20dBm maximum
\item Receiver sensitivity: -148dBm
\item Range: Up to 15km (line of sight)
\item Power consumption: 10.8mA TX, 10.3mA RX
\end{itemize}

\section{Power Supply Circuits}
\label{sec:power_supply_circuits}

\subsection{Main Power Supply Circuit}
\label{subsec:main_power_supply}

The main power supply provides stable DC voltages for all system components with protection against over-voltage and over-current conditions.

\begin{figure}[H]
\centering
\includegraphics[width=0.9\textwidth]{power_supply_circuit}
\caption{Main Power Supply Circuit Diagram}
\label{fig:power_supply_circuit}
\end{figure}

\textbf{Circuit Components:}
\begin{itemize}
\item \textbf{T2:} Step-down transformer (230V:15V, 10VA)
\item \textbf{D3-D6:} Bridge rectifier diodes (1N4007)
\item \textbf{C16:} Filter capacitor (2200$\mu$F, 25V)
\item \textbf{U6:} 5V linear regulator (LM7805)
\item \textbf{U7:} 3.3V low-dropout regulator (AMS1117-3.3)
\item \textbf{C17-C20:} Output filter capacitors
\item \textbf{F1:} Input fuse (500mA, slow-blow)
\item \textbf{L4:} Common mode choke (100$\mu$H)
\end{itemize}

\textbf{Design Specifications:}
\begin{itemize}
\item Input voltage: 85-265V AC, 47-63Hz
\item Output voltages: +5V @ 1A, +3.3V @ 500mA
\item Regulation: $\pm$5\% line and load
\item Ripple voltage: <50mV peak-to-peak
\item Efficiency: >75\% at full load
\item Protection: Over-current, over-temperature
\end{itemize}

\subsection{Battery Backup Circuit}
\label{subsec:battery_backup_circuit}

The battery backup circuit ensures continuous operation during power outages and provides power management for portable sensor nodes.

\begin{figure}[H]
\centering
\includegraphics[width=0.9\textwidth]{battery_backup_circuit}
\caption{Battery Backup Circuit Diagram}
\label{fig:battery_backup_circuit}
\end{figure}

\textbf{Circuit Components:}
\begin{itemize}
\item \textbf{BAT1:} 12V sealed lead-acid battery (7Ah)
\item \textbf{U8:} Battery charge controller (LTC4162)
\item \textbf{Q1, Q2:} Power switching MOSFETs (IRF540N)
\item \textbf{D7, D8:} Schottky diodes for reverse polarity protection
\item \textbf{R18, R19:} Current sensing resistors (0.1$\Omega$, 1W)
\item \textbf{U9:} Battery monitoring IC (LTC2944)
\item \textbf{LED1, LED2:} Status indication LEDs
\end{itemize}

\textbf{Design Specifications:}
\begin{itemize}
\item Battery type: 12V sealed lead-acid (SLA)
\item Charging current: 0.1C to 0.3C (adjustable)
\item Float voltage: 13.6V $\pm$ 1\%
\item Backup time: 8-12 hours (depending on load)
\item Monitoring: Voltage, current, capacity, temperature
\item Protection: Over-charge, over-discharge, short-circuit
\end{itemize}

\section{Data Acquisition and Processing Circuits}
\label{sec:data_acquisition_circuits}

\subsection{Multi-channel ADC Circuit}
\label{subsec:adc_circuit}

The multi-channel ADC circuit provides high-resolution analog-to-digital conversion for simultaneous sampling of multiple sensor inputs.

\begin{figure}[H]
\centering
\includegraphics[width=0.9\textwidth]{adc_circuit}
\caption{Multi-channel ADC Circuit Diagram}
\label{fig:adc_circuit}
\end{figure}

\textbf{Circuit Components:}
\begin{itemize}
\item \textbf{U10:} 16-bit, 8-channel ADC (ADS1115)
\item \textbf{R20-R23:} Input scaling resistors (10k$\Omega$, 0.1\%)
\item \textbf{C21-C24:} Anti-aliasing filter capacitors (1nF, NPO)
\item \textbf{R24, R25:} I2C pull-up resistors (4.7k$\Omega$)
\item \textbf{C25:} Reference bypass capacitor (100nF)
\end{itemize}

\textbf{Design Specifications:}
\begin{itemize}
\item Resolution: 16 bits
\item Sampling rate: Up to 860 SPS
\item Input channels: 8 single-ended or 4 differential
\item Input voltage range: 0-5.5V (programmable gain)
\item Interface: I2C (up to 3.4MHz)
\item Accuracy: $\pm$2 LSB INL, $\pm$1 LSB DNL
\end{itemize}

\subsection{Signal Conditioning Circuit}
\label{subsec:signal_conditioning}

The signal conditioning circuit provides filtering, amplification, and level shifting for various sensor signals before digitization.

\begin{figure}[H]
\centering
\includegraphics[width=0.9\textwidth]{signal_conditioning_circuit}
\caption{Signal Conditioning Circuit Diagram}
\label{fig:signal_conditioning_circuit}
\end{figure}

\textbf{Circuit Components:}
\begin{itemize}
\item \textbf{U11:} Quad operational amplifier (LM324)
\item \textbf{R26-R33:} Gain setting and bias resistors
\item \textbf{C26-C29:} Filter capacitors for anti-aliasing
\item \textbf{D9, D10:} Input protection diodes
\item \textbf{RV1, RV2:} Trim potentiometers for calibration
\end{itemize}

\textbf{Design Specifications:}
\begin{itemize}
\item Gain range: 1 to 100 (software programmable)
\item Bandwidth: DC to 10kHz (-3dB)
\item Input impedance: >10M$\Omega$
\item Output impedance: <100$\Omega$
\item Common mode rejection: >80dB
\item Power supply rejection: >80dB
\end{itemize}

\section{PCB Layout Considerations}
\label{sec:pcb_layout}

\subsection{General Layout Guidelines}
\label{subsec:layout_guidelines}

The PCB layout follows industry best practices for mixed-signal designs:

\textbf{Layer Stack-up:}
\begin{itemize}
\item Layer 1: Component placement and routing
\item Layer 2: Ground plane
\item Layer 3: Power plane (+3.3V, +5V)
\item Layer 4: Signal routing and component placement
\end{itemize}

\textbf{Design Rules:}
\begin{itemize}
\item Minimum trace width: 0.1mm (4 mil)
\item Minimum via size: 0.2mm (8 mil)
\item Minimum drill size: 0.15mm (6 mil)
\item Minimum spacing: 0.1mm (4 mil)
\item Copper thickness: 35$\mu$m (1 oz)
\end{itemize}

\subsection{RF Layout Considerations}
\label{subsec:rf_layout}

Special attention is given to RF circuit layout for optimal performance:

\textbf{Antenna Placement:}
\begin{itemize}
\item Antennas placed at board edges with keepout zones
\item Minimum 5mm clearance from other components
\item Ground plane cutouts for antenna radiation
\end{itemize}

\textbf{RF Trace Routing:}
\begin{itemize}
\item 50$\Omega$ controlled impedance for single-ended signals
\item 100$\Omega$ differential impedance for balanced signals
\item Minimum bend radius of 3× trace width
\item Via stitching for ground plane continuity
\end{itemize}

\subsection{Power and Ground Distribution}
\label{subsec:power_ground}

Proper power and ground distribution is critical for system performance:

\textbf{Power Distribution:}
\begin{itemize}
\item Dedicated power planes for each supply voltage
\item Power plane splits at analog/digital boundaries
\item Ferrite beads for supply noise isolation
\item Local decoupling capacitors at each IC
\end{itemize}

\textbf{Ground Distribution:}
\begin{itemize}
\item Single-point ground connection between analog and digital
\item Star ground distribution for sensitive analog circuits
\item Ground plane stitching vias every 5mm
\item Guard rings around sensitive circuits
\end{itemize}

\section{Circuit Analysis and Simulation}
\label{sec:circuit_analysis}

\subsection{SPICE Simulation Results}
\label{subsec:spice_simulation}

Circuit simulation using SPICE models validates the design performance:

\textbf{Voltage Sensor Transfer Function:}
\begin{itemize}
\item DC gain: 0.008V/V (accurate voltage scaling)
\item Bandwidth: 1.2kHz (-3dB point)
\item Phase response: <5° phase shift at 50Hz
\item Total harmonic distortion: <0.1\% at rated input
\end{itemize}

\textbf{Current Sensor Frequency Response:}
\begin{itemize}
\item Gain accuracy: $\pm$0.2\% from 45-65Hz
\item Phase accuracy: $\pm$0.5° from 45-65Hz
\item Common mode rejection: >90dB at 50Hz
\item Input impedance: >1M$\Omega$ (transformer secondary)
\end{itemize}

\subsection{Thermal Analysis}
\label{subsec:thermal_analysis}

Thermal analysis ensures reliable operation under all conditions:

\textbf{Component Temperature Rise:}
\begin{itemize}
\item Linear regulators: 35°C rise at maximum load
\item Power MOSFETs: 40°C rise during switching
\item RF components: 25°C rise at maximum transmission power
\item Critical components operate within specifications
\end{itemize}

\textbf{Cooling Requirements:}
\begin{itemize}
\item Natural convection adequate for most components
\item Heat sinks required for high-power linear regulators
\item Thermal vias used for heat spreading
\item Operating temperature range: -20°C to +70°C
\end{itemize}

\section{Component Selection Rationale}
\label{sec:component_selection}

\subsection{Active Components}
\label{subsec:active_components}

Component selection is based on performance requirements, cost, and availability:

\textbf{Microcontrollers and Processors:}
\begin{itemize}
\item ESP32: Integrated WiFi/Bluetooth, low cost, good development support
\item ARM Cortex-M series: High performance, low power, extensive ecosystem
\item Raspberry Pi: Linux support, GPIO capabilities, community support
\end{itemize}

\textbf{Analog Components:}
\begin{itemize}
\item LM358: General-purpose op-amp, low cost, single supply operation
\item INA128: High precision instrumentation amplifier, excellent CMRR
\item ADS1115: High resolution ADC, I2C interface, programmable gain
\end{itemize}

\subsection{Passive Components}
\label{subsec:passive_components}

Passive component selection focuses on precision and stability:

\textbf{Resistors:}
\begin{itemize}
\item Metal film resistors for precision applications (0.1\% tolerance)
\item Carbon film resistors for general applications (5\% tolerance)
\item Current sensing resistors with low temperature coefficient
\end{itemize}

\textbf{Capacitors:}
\begin{itemize}
\item Ceramic capacitors for decoupling (X7R dielectric)
\item Tantalum capacitors for power supply filtering
\item Film capacitors for RF applications (low ESR, stable)
\end{itemize}

\section{Testing and Validation Results}
\label{sec:testing_results}

\subsection{Laboratory Test Results}
\label{subsec:lab_test_results}

Comprehensive laboratory testing validates circuit performance:

\textbf{Accuracy Testing:}
\begin{itemize}
\item Voltage measurement accuracy: $\pm$0.8\% of reading
\item Current measurement accuracy: $\pm$0.5\% of reading
\item Temperature measurement accuracy: $\pm$0.3°C
\item Power calculation accuracy: $\pm$1.2\% of reading
\end{itemize}

\textbf{Stability Testing:}
\begin{itemize}
\item 24-hour drift test: <0.1\% variation
\item Temperature coefficient: <50ppm/°C
\item Long-term stability: <0.2\% over 1000 hours
\item Calibration interval: 12 months recommended
\end{itemize}

\subsection{EMC Testing Results}
\label{subsec:emc_testing}

Electromagnetic compatibility testing ensures regulatory compliance:

\textbf{Emissions Testing:}
\begin{itemize}
\item Conducted emissions: Meets CISPR 22 Class B limits
\item Radiated emissions: 6dB margin below FCC Part 15 limits
\item Spurious emissions: >40dB below carrier level
\end{itemize}

\textbf{Immunity Testing:}
\begin{itemize}
\item ESD immunity: ±4kV contact, ±8kV air discharge
\item RF immunity: 10V/m field strength, 80MHz-1GHz
\item Burst immunity: ±2kV on power lines, ±1kV on signal lines
\item Surge immunity: ±2kV on power lines
\end{itemize}

\section{Manufacturing and Assembly}
\label{sec:manufacturing}

\subsection{PCB Fabrication Specifications}
\label{subsec:pcb_fabrication}

\textbf{PCB Specifications:}
\begin{itemize}
\item Board thickness: 1.6mm (standard)
\item Copper weight: 1oz (35$\mu$m) outer layers, 0.5oz inner layers
\item Surface finish: HASL (Hot Air Solder Leveling)
\item Solder mask: Green, LPI (Liquid Photo Imageable)
\item Silkscreen: White, both sides
\item IPC Class: IPC-A-600 Class 2 (General Electronic Products)
\end{itemize}

\textbf{Fabrication Tolerances:}
\begin{itemize}
\item Hole position accuracy: $\pm$0.05mm
\item Track width tolerance: $\pm$10\%
\item Via registration: $\pm$0.05mm
\item Impedance control: $\pm$10\%
\end{itemize}

\subsection{Assembly Process}
\label{subsec:assembly_process}

\textbf{Surface Mount Assembly:}
\begin{itemize}
\item Solder paste application using stencil printing
\item Component placement using pick-and-place machine
\item Reflow soldering using controlled temperature profile
\item Automated optical inspection (AOI) for quality control
\end{itemize}

\textbf{Through-hole Assembly:}
\begin{itemize}
\item Selective wave soldering for through-hole components
\item Manual soldering for special components
\item Final inspection and functional testing
\item Conformal coating for environmental protection
\end{itemize}

\textbf{Quality Control:}
\begin{itemize}
\item In-circuit testing (ICT) for component verification
\item Functional testing with automated test equipment
\item Calibration and trimming procedures
\item Final packaging and documentation
\end{itemize}